% Options for packages loaded elsewhere
\PassOptionsToPackage{unicode}{hyperref}
\PassOptionsToPackage{hyphens}{url}
\PassOptionsToPackage{dvipsnames,svgnames,x11names}{xcolor}
%
\documentclass[
]{article}

\usepackage{amsmath,amssymb}
\usepackage{iftex}
\ifPDFTeX
  \usepackage[T1]{fontenc}
  \usepackage[utf8]{inputenc}
  \usepackage{textcomp} % provide euro and other symbols
\else % if luatex or xetex
  \usepackage{unicode-math}
  \defaultfontfeatures{Scale=MatchLowercase}
  \defaultfontfeatures[\rmfamily]{Ligatures=TeX,Scale=1}
\fi
\usepackage{lmodern}
\ifPDFTeX\else  
    % xetex/luatex font selection
\fi
% Use upquote if available, for straight quotes in verbatim environments
\IfFileExists{upquote.sty}{\usepackage{upquote}}{}
\IfFileExists{microtype.sty}{% use microtype if available
  \usepackage[]{microtype}
  \UseMicrotypeSet[protrusion]{basicmath} % disable protrusion for tt fonts
}{}
\makeatletter
\@ifundefined{KOMAClassName}{% if non-KOMA class
  \IfFileExists{parskip.sty}{%
    \usepackage{parskip}
  }{% else
    \setlength{\parindent}{0pt}
    \setlength{\parskip}{6pt plus 2pt minus 1pt}}
}{% if KOMA class
  \KOMAoptions{parskip=half}}
\makeatother
\usepackage{xcolor}
\setlength{\emergencystretch}{3em} % prevent overfull lines
\setcounter{secnumdepth}{5}
% Make \paragraph and \subparagraph free-standing
\ifx\paragraph\undefined\else
  \let\oldparagraph\paragraph
  \renewcommand{\paragraph}[1]{\oldparagraph{#1}\mbox{}}
\fi
\ifx\subparagraph\undefined\else
  \let\oldsubparagraph\subparagraph
  \renewcommand{\subparagraph}[1]{\oldsubparagraph{#1}\mbox{}}
\fi


\providecommand{\tightlist}{%
  \setlength{\itemsep}{0pt}\setlength{\parskip}{0pt}}\usepackage{longtable,booktabs,array}
\usepackage{calc} % for calculating minipage widths
% Correct order of tables after \paragraph or \subparagraph
\usepackage{etoolbox}
\makeatletter
\patchcmd\longtable{\par}{\if@noskipsec\mbox{}\fi\par}{}{}
\makeatother
% Allow footnotes in longtable head/foot
\IfFileExists{footnotehyper.sty}{\usepackage{footnotehyper}}{\usepackage{footnote}}
\makesavenoteenv{longtable}
\usepackage{graphicx}
\makeatletter
\def\maxwidth{\ifdim\Gin@nat@width>\linewidth\linewidth\else\Gin@nat@width\fi}
\def\maxheight{\ifdim\Gin@nat@height>\textheight\textheight\else\Gin@nat@height\fi}
\makeatother
% Scale images if necessary, so that they will not overflow the page
% margins by default, and it is still possible to overwrite the defaults
% using explicit options in \includegraphics[width, height, ...]{}
\setkeys{Gin}{width=\maxwidth,height=\maxheight,keepaspectratio}
% Set default figure placement to htbp
\makeatletter
\def\fps@figure{htbp}
\makeatother
% definitions for citeproc citations
\NewDocumentCommand\citeproctext{}{}
\NewDocumentCommand\citeproc{mm}{%
  \begingroup\def\citeproctext{#2}\cite{#1}\endgroup}
\makeatletter
 % allow citations to break across lines
 \let\@cite@ofmt\@firstofone
 % avoid brackets around text for \cite:
 \def\@biblabel#1{}
 \def\@cite#1#2{{#1\if@tempswa , #2\fi}}
\makeatother
\newlength{\cslhangindent}
\setlength{\cslhangindent}{1.5em}
\newlength{\csllabelwidth}
\setlength{\csllabelwidth}{3em}
\newenvironment{CSLReferences}[2] % #1 hanging-indent, #2 entry-spacing
 {\begin{list}{}{%
  \setlength{\itemindent}{0pt}
  \setlength{\leftmargin}{0pt}
  \setlength{\parsep}{0pt}
  % turn on hanging indent if param 1 is 1
  \ifodd #1
   \setlength{\leftmargin}{\cslhangindent}
   \setlength{\itemindent}{-1\cslhangindent}
  \fi
  % set entry spacing
  \setlength{\itemsep}{#2\baselineskip}}}
 {\end{list}}
\usepackage{calc}
\newcommand{\CSLBlock}[1]{\hfill\break\parbox[t]{\linewidth}{\strut\ignorespaces#1\strut}}
\newcommand{\CSLLeftMargin}[1]{\parbox[t]{\csllabelwidth}{\strut#1\strut}}
\newcommand{\CSLRightInline}[1]{\parbox[t]{\linewidth - \csllabelwidth}{\strut#1\strut}}
\newcommand{\CSLIndent}[1]{\hspace{\cslhangindent}#1}

\usepackage[noblocks]
{authblk}
\renewcommand*{\Authsep}{, }
\renewcommand*{\Authand}{, }
\renewcommand*{\Authands}{, }
\renewcommand\Affilfont{\small}
\DeclareMathOperator*{\argmin}{arg\,min}
\makeatletter
\@ifpackageloaded{caption}{}{\usepackage{caption}}
\AtBeginDocument{%
\ifdefined\contentsname
  \renewcommand*\contentsname{Table of contents}
\else
  \newcommand\contentsname{Table of contents}
\fi
\ifdefined\listfigurename
  \renewcommand*\listfigurename{List of Figures}
\else
  \newcommand\listfigurename{List of Figures}
\fi
\ifdefined\listtablename
  \renewcommand*\listtablename{List of Tables}
\else
  \newcommand\listtablename{List of Tables}
\fi
\ifdefined\figurename
  \renewcommand*\figurename{Figure}
\else
  \newcommand\figurename{Figure}
\fi
\ifdefined\tablename
  \renewcommand*\tablename{Table}
\else
  \newcommand\tablename{Table}
\fi
}
\@ifpackageloaded{float}{}{\usepackage{float}}
\floatstyle{ruled}
\@ifundefined{c@chapter}{\newfloat{codelisting}{h}{lop}}{\newfloat{codelisting}{h}{lop}[chapter]}
\floatname{codelisting}{Listing}
\newcommand*\listoflistings{\listof{codelisting}{List of Listings}}
\makeatother
\makeatletter
\makeatother
\makeatletter
\@ifpackageloaded{caption}{}{\usepackage{caption}}
\@ifpackageloaded{subcaption}{}{\usepackage{subcaption}}
\makeatother
\ifLuaTeX
  \usepackage{selnolig}  % disable illegal ligatures
\fi
\usepackage{bookmark}

\IfFileExists{xurl.sty}{\usepackage{xurl}}{} % add URL line breaks if available
\urlstyle{same} % disable monospaced font for URLs
\hypersetup{
  pdftitle={Measuring the economic reasoning abilities of language models},
  pdfauthor={Douglas K. G. Araujo},
  colorlinks=true,
  linkcolor={blue},
  filecolor={Maroon},
  citecolor={Blue},
  urlcolor={Blue},
  pdfcreator={LaTeX via pandoc}}

\title{Measuring the economic reasoning abilities of language
models\thanks{This work represents my opinion and not necessarily that
of the BIS.}}


  \author{Douglas K. G. Araujo}
            \affil{%
                  Bank for International
Settlements, douglas.araujo@bis.org
              }
      
\date{}
\begin{document}
\maketitle
\begin{abstract}
Economic reasoning is represented as a state space function.
\end{abstract}

\section{Introduction}\label{introduction}

Language models (LMs), in particular those classified as generative
artificial intelligence (gen AI), are finding increasing uses in finance
and economics. These models are usually tested for their ability to
reason, and seem to do well: for example, OpenAI's GPT-4 boats more than
80\% correct results in academic and professional micro- and
macroeconomics tests (Achiam et al. (2023)). Still, even such advanced
models can fail miserably. Perez-Cruz and Shin (2024) demonstrate how
the same model can correctly solve a logical puzzle requiring reasoning
about higher order knowledge, only to fail when irrelevant details are
changed. Building on results such as this and other examples that
clearly illustrate the limits of rationality assumptions on LMs, this
work discusses how to systematically measure \emph{economic} reasoning,
combining literatures on economic thought and on computer science about
gen AI benchmarking. In practical terms, the task at hand is to come up
with testing mechanisms that estimate the level of economic reasoning of
an LM by means of a prompt consisting of \(n \geq 0\) examples and a
question with multiple answers.

At its most essential form, testing for economic reasoning is the same
as probing if the model is able to think in terms of logical operators.
However, they can be subjective because (a) economic thought is always
changing and (b) they are only as good as their abilites ot explain
limited sets of reality (those that modern academics constantly see=
rather than any other reality).

Similar to many other social disciplines, economics requires the
analytical judgment referred to by Robbins (1932) in the analyses of
events as a basis to extrapolate and predict, and this has a bearing on
how economic reasoning should be benchmarked. Economic inference depends
primarily on articulating unobservable quantities, theorecised and
estimated on the basis of observable measures. This is unlike other
major disciplines. For example, in human and veterinary medicine, all
physiological and pathological variables of clinical importance are
observable, even if that is not yet technologically feasible today. In
the medical sciences, theoretical models merely fill in the gaps in the
absence of a technologically feasible complete measurement. In contrast,
many economically relevant quantities are latent variables that cannot
by definition be observed, and always require a model applied to data to
be estimated, implicit or not.

A quantitative test for economic reasoning must take this into account:
selecting a correct answer in an economics question through reasoning
will always depend on an unobserved transformation of the information
received and the existing knowledge. This is important. LMs may also
happen to choose the correct answer from either luck of through simple
token probability. It is easy to see why a correct answer selected by
chance is not informative about the reasoning abilities of a model. The
second case requires more explanation: mathematically, LMs are trained
to identify the most likely token \(\theta\) in a vocabulary \(V\) given
the tokens in its prompt. In practice, the function is inexcrutable so
it is also considered an unobservable transformation. But a few
characteristics allows us to distinguish reasoning from prediction.
First, reasoning is robust to minutiae and other irrelevant detail.
Mathematically, it would be analogous to applying a manifold
transformation that retains only the relevant information in a prompt
and then applies logic operations on top of them, and on them only.
Second, reasoning is locally complete, meaning that an LM that can
correctly deduce that A implies B also is able to understand that A'
does not imply B, or that A does not imply B'. In other words, a
reasoning that appears to be correct but whose obvious corolary is not
achieved by an LM cannot be said to have been reasoned in the first
place.

Knowledge: linguistic, common and commonsense.

Interpretation. information theory. Shannon.

The main intuition of this work is to combine a number of building
blocks of evaluation.

\begin{itemize}
\item
  the benchmark must be challenging for machines: I use an adjusted
  version of adversarial filtering (Zellers et al. (2019)) to create
  answer candidates that are hard for LMs to guess
\item
  the test must incorporate slow-moving evolutions in academic economic
  thought: evolving test set based on newly published academic work.
\item
  results related to reasoning must be distinguished as best as possible
  from the ability to interpret the prompt or from knowledge (implicit
  or explicit) about economics, ie reasoning is a separate step: sets of
  perturbations in the spirit of Alzahrani et al. (2024) for each
  initial task.
\end{itemize}

the benchmark counts with a mathematical adjustment that takes into
account performance across perturbations, penalising results that vary
with \ldots.

This benchmark evaluation addresses a poignant issue for the economics
profession: the lack of publicly available data about how these
benchmarks are created and any, and toasted.

A major inspiration in the design of the questions and how they can
generate identifying variations is the social economics literature. A
key reference is Stantcheva (2023). The idea here is that the design of
the questionnaire itself can elicit responses that allow for insight
into non-observable traits such as reasoning. Many of the insights of
this literature carry over naturally to the machine space.\footnote{Actually
  testing whether LMs \emph{do not} parrot or ``organically'' exhibit
  biases or other behaviours that are assumed to be exclusively human
  would be an interesting line of research.}

A substantial body of work creates and discusses benchmarking models in
general. A very useful reference is Storks, Gao, and Chai (2020).
Literature on benchmarking economic reasoning appears to be new,
although other works have touched upon the topic from different angles.
An early foray into questions related to AI's ability to conduct
economic reasoning is due to Parkes and Wellman (2015). But their angle
is more on how AIs can be used to estimate synthetic economic agents -
machina oeconomicus - ideal versions of purely rational agents, rather
than on the measurement and the implications of AIs acquiring economic
reasoning abilities. In any case, Parkes and Wellman (2015) see economic
reasoning as the ability to understand and solve complex
game-theoretical environments (eg, the poker example).

\section{Lessons from human surveys}\label{lessons-from-human-surveys}

I use a considerable amount of specific advice on human surveys from
Stantcheva (2023) to generate identifying variation in the questions.
Specifically:

\begin{itemize}
\tightlist
\item
  coeteris paribus questions
\item
  pre-testing
\item
  including possibilities for blank, indifferent or even recognise that
  AI does not know
\item
  avoiding jargon
\item
  questions that check for ``attention'' and ``effort'' on the part of
  the respondent
\item
  also including open ended questions (as in Ferrario and Stantcheva
  (2022))

  \begin{itemize}
  \tightlist
  \item
    including follow-up questions (``are thre any other reasons'')
  \item
    going beyond Ferrario and Stantcheva (2022), in this paper I use
    open ended questions that are similar in nature to closed end
    questions and deploy large language models to interpret them.
  \end{itemize}
\item
  question ordering

  \begin{itemize}
  \tightlist
  \item
    in particular, consideration is given to whether each question
    should be presented to a separate instance of the LM, or the full
    questionnaire could be shared in the same ``chat''.
  \end{itemize}
\item
  take due consideration of how to address the different types of bias
  associated with surveys (adapted for the machine context, naturally)
\end{itemize}

\section{Desirable characteristics of a
benchmark}\label{desirable-characteristics-of-a-benchmark}

\subsection{Evolve over time}\label{evolve-over-time}

Economic reasoning evolves over time. For example, the Lucas critique
(Lucas (1976)) was influential in shifting macroeconomic modelling,
while the credibility revolution described in Angrist and Pischke (2008)
was similarly influential in microeconomic work.

\section{A model of economic
reasoning}\label{a-model-of-economic-reasoning}

The result from existing benchmarks is largely, if not completely,
directly related to the number of questions correctly answered. However,
this measures only the model's ability to answer correctly, \emph{not
necessarily} its reasoning capabilities. The latter are part of a latent
state space sitting between the input prompt and the answer. More
concretely, for an input prompt \(X\), which includes a question and any
necessary explicit information, the language model is a function
\(\mathbf{M}\) that maps it to a given response:
\(\mathbf{M} : X \to y\). In order to show that it is done by reasoning,
we need tests (and more specifically, measurements) that convey some
information about the inner workings of this function.

\subsection{Reasoning as an abstract of the
input}\label{reasoning-as-an-abstract-of-the-input}

\begin{itemize}
\item
  Input prompt \(X\)
\item
  Transformed into \(g(X, \kappa)\), a state space function that also
  takes the existing knowledge \(\kappa\) and associates it with the
  prompt to maps it to its abstract fundamentals (similar to manifold
  learning)
\item
  Result based on \(g(X)\).
\end{itemize}

\subsection{A (very) simple model}\label{a-very-simple-model}

This section builds on the intuition that in true reasoning, the result
should be robust to minute perturbations, ie the model is a constant
function over the domain of the input. Formally, both
\(\mathbf{M}(X) = y\) and \(\mathbf{M}(X + \epsilon) = y\) for an
infinitesimal \(\epsilon\). This implies the derivative with respect to
the input prompt is zero. Using as an approachable example the simplest
possible neural network, the logistic regression
\(\mathbf{N}(x) = \sigma(Wx + b)\), such robustness further implies that
\(\frac{d\mathbf{N}}{d x} = \sigma(Wx + b)(1-\sigma(Wx + b))W = 0\).
Because \(W\) cannot be a zero vector in a functioning network that is
responsive to its inputs and \(\sigma(Wx + b)(1-\sigma(Wx + b)) = 0\)
has no solution because neither term is 0 or 1 in a sigmoid function
with finite inputs, the neural network cannot be a constant function.
This extremely simplified example, which holds for recursive
architectures of similarly simple layers, does not bode well for the
robustness of results given small perturbations in the input prompt.

\section{Reasoning benchmarks in other
fields}\label{reasoning-benchmarks-in-other-fields}

\begin{itemize}
\item
  Math
\item
  Medical
\item
  Biologia
\end{itemize}

\section{A model of reasoning}\label{a-model-of-reasoning}

This section develops a model of reasoning that fits naturally into both
natural and artificial LMs. It will serve as the basis for the
subsequent analyses and empirical creation of a reasoning benchmark.

Let a sentence \(\mathbf{S} = (\theta_1, \theta_2, \theta_3, ...)\) be a
sequence of token-location tuples \(\theta_x = (\tau, x)\), with each
\(\tau \in \mathbf{V}\) belonging to a vocabulary \(\mathbf{V}\) and
\(x \in \mathbb{N}^{d_{\text{model}}}\).\footnote{The location is
  important because it helps define meaning, along with the actual
  letter (more generally, symbol) content of th token. Note that in this
  paper, white spaces are abstracted away for expositional simplicity.}
Create a function \(\pi_{i, C} : \theta, \mathbf{S} \to \{-1, 0, 1\}\)
that maps each token into one of three possibilities: the token's
information can be considered a adversarial (-1), irrelevant (0) or
relevant (1) with respect to the likelihood of individual (or LM) \(i\)
uttering another sentence C. For example, take the following quote from
the character Barf in the 1987 movie Spaceballs, organised as two
sentences ``I'm a mog. Half man, half dog.'' and ``I'm my own best
friend.'' With word-level tokenisation,
\(\mathbf{S} = \{("\text{I'm}", 1), ("\text{a}", 2), ("\text{mog}", 3), ("\text{.}", 4), ("\text{Half}", 5), ("\text{man}", 6), ("\text{,}", 7), ("\text{half}", 8), ("\text{dog}", 9), ("\text{.}", 10)\}\)
and \(\mathbf{C}\) is similarly broken down. This example illustrates
that even when there is not a logical connection grounded in truth,
tokens in one sentence - even those made up like ``mog'', can have a
bearing on the likelihood of tokens appearing in another sentence. This
likelihood can differ depending on the location of the token, which also
allows for situations where repeteating of a word \(\tau\) is meant to
convey different meaning. Another feature of this example is that all
\(\pi_{\text{Barf}, C}(\theta) = 1 \forall \theta \in \mathbf{S}\). In
the alternative sentence ``I'm a mog. Half man, half dog. I am alive.'',
the new component is obviously irrelevant for \(\mathbf{C}\):
\(\prod_{x \in [10, 14]} \pi_{\text{Barf}, C}(\theta_x) = 0\).

This exposition is important to delve into the reasoning aspect,
entirely organised by function \(\pi\). Since \(\pi_{i, C}\) measures
how informative a token is for individual \(i\)'s \(\mathbf{C}\), it
constitutes the first aspect of reasoning: to recognise when a token is
adversarial, irrelevant or relevant. This step is necessary before the
application of any logical rules \(\mathcal{l} \in \mathcal{L}\) on the
weighted token, \(\pi_{i, C}(\theta_x) \theta_x\). The exact
underpinnings of these logical rules are beyond the scope of this work -
it can be approximated by a possibly non-linear function, \(g\). What
suffices in this work is to say that reasoning \emph{depends} on
correctly classifying the tokens: all relevant tokens must be so
identified, lest they be either ignored as the irrelevant ones or taken
with the opposite meaning. Similarly, if all relevant tokens are indeed
diagnosed correctly but other tokens are also diagnosed as relevant when
they are not, then this will cause problems for the correct reasoning.
In other words, a first precondition for reasoning is to have a low
categorical cross-entropy loss. Intuitively, a pre-condition of
reasoning is to correctly interpret the inputs.

Use Taylor expansion on model since its derivative to perturbation
should be zero. This gives us a head start in the Taylor expansion. Try
to link the T-expanded equation to an estimating equation.

But what determines \(\pi_{i, C}\)? A combination of knowledges and
logical relationships.

Knowledges: linguistic knowledge, common knowledge and commonsense
knowledge

Rationales: reasoning from logic

Armed with the sentence-level categorical cross-entropy, the individual
can establish chains of thought that will finally lead to reasoning.
Again, for simplicity, the exact function is not discussed here, other
than that it is a potentially simple or complex way to interact. What is
important is to add the categorical cross-entropy to the estimation
equation.

\textbf{Benchmark testing mechanism}\ldots{}

\section{Reasoning about economics}\label{reasoning-about-economics}

The model above allows us to estimate reasoning while also breaking down
some of its components to better understand them. For example, we can
estimate any errors in reasoning into an issue with
\textbf{interpretation}, \textbf{knowledge} and \textbf{logical
thinking}. The empirical estimation follows.

\section{Empirical estimation}\label{empirical-estimation}

Each \emph{task} \(\theta \in \Theta\) can be asked in various different
ways, each one being called a \emph{question} \(q \in \theta\).
Questions vary with respect to their adversarial aspect; it is this
variation within each question that allows the empirical estimation of
the effects associated with interpretation or with knowledge. Most of
the variations are originally those tested in Alzahrani et al. (2024).
The variation in response between the questions within each task will
comprise the evaluation of the actual reasoning capabilities. As alluded
to before, the variations are organised into those that measure the
stability of a response to adversarial interpretation answers, and those
that measure the stability across the knowledge dimension. In practice,
each task has hundreds of different \(q\). These groups are described in
more detail next.

\subsection{Variations related to
interpretation}\label{variations-related-to-interpretation}

There are several classes of variations that can help test an LMs'
interpretation.

\subsubsection{Choice variations}\label{choice-variations}

Here the choices remain the same for a task but vary in their order
across questions

\begin{itemize}
\item
  random choice order
\item
  biased choice order
\item
  uncommon answer choice symbols
\item
  common but unordered answer choice symbols
\end{itemize}

\subsubsection{Word variations}\label{word-variations}

The main idea here is to introduce or change words that are irrelevant.
This is along the lines of the test conducted by Perez-Cruz and Shin
(2024).

Another one is to conduct random word repetition as if it were a typo

\subsection{Variations related to
knowledge}\label{variations-related-to-knowledge}

Changing key words related to field knowledge with other field knowledge
words but that would not make a sense to an expert. This can be compared
with just changing the same words into another generic word. Comparing
responses between both should indicate the level of knowledge used by
the model (should it? need to think more)

\subsection{Estimation formula}\label{estimation-formula}

The main formula is akin to the linear probability model since \(a_{q}\)
is either zero or one:

\[
a_{q} = \beta_{\theta} \theta + \beta_{\text{Interpretation}} \eta_q + \beta_{\text{Knowledge}} \kappa_q + \epsilon_q
\]

Another idea to explore is whether these variations can actually
instrument interpretation and knowledge. This would allow the formula to
estimate the reasoning bit.

\section{Operational characteristics}\label{operational-characteristics}

\begin{itemize}
\tightlist
\item
  avoid becoming part of training data
\end{itemize}

Some drawbacks of using academic papers include:

\begin{itemize}
\tightlist
\item
  bias to report only positive findings (and to do so in a way that is
  generous towards said findings)
\item
  Also, academic papers suffer from false negatives: many contributions
  that are now considered classics have been previously rejected (Gans
  and Shepherd (1994)).
\end{itemize}

\section{Conclusions}\label{conclusions}

As economic agents and policymakers harness generative artificial
intelligence (AI) to reap considerable efficiencies, and thus their
societal footprint becomes larger, a benchmark for economic reasoning is
needed. I suggest ways to implement such a benchmark, and measure the
current performance of a selected list of LMs.

\section{Annex 1: discussion of biases in human surveys and how they
could affect LM
questionnaires}\label{annex-1-discussion-of-biases-in-human-surveys-and-how-they-could-affect-lm-questionnaires}

\begin{itemize}
\tightlist
\item
  Section A-4 in Stantcheva (2023)
\end{itemize}

The goal of this annex is to list side-by-side the main human biases
that affect survey responses and their corresponding machine version, if
any (from a theoretical perspective - it would be interesting to test if
LMs carry over some of these biases that are supposed to be only human,
which could suggest they are parroting or in extremis developing sources
of bias like shame, etc).

\section*{References}\label{references}
\addcontentsline{toc}{section}{References}

\phantomsection\label{refs}
\begin{CSLReferences}{1}{0}
\bibitem[\citeproctext]{ref-achiam2023gpt}
Achiam, Josh, Steven Adler, Sandhini Agarwal, Lama Ahmad, Ilge Akkaya,
Florencia Leoni Aleman, Diogo Almeida, et al. 2023. {``Gpt-4 Technical
Report.''} \emph{arXiv Preprint arXiv:2303.08774}.

\bibitem[\citeproctext]{ref-alzahrani2024benchmarks}
Alzahrani, Norah, Hisham Abdullah Alyahya, Yazeed Alnumay, Sultan
Alrashed, Shaykhah Alsubaie, Yusef Almushaykeh, Faisal Mirza, et al.
2024. {``When Benchmarks Are Targets: Revealing the Sensitivity of Large
Language Model Leaderboards.''} \emph{arXiv Preprint arXiv:2402.01781}.

\bibitem[\citeproctext]{ref-angrist2008mostly}
Angrist, Joshua D., and Jörn-Steffen Pischke. 2008. \emph{Mostly
Harmless Econometrics: An Empiricist's Companion}. Princeton University
Press.

\bibitem[\citeproctext]{ref-ferrario2022eliciting}
Ferrario, Beatrice, and Stefanie Stantcheva. 2022. {``Eliciting People's
First-Order Concerns: Text Analysis of Open-Ended Survey Questions.''}
In \emph{AEA Papers and Proceedings}, 112:163--69. American Economic
Association 2014 Broadway, Suite 305, Nashville, TN 37203.

\bibitem[\citeproctext]{ref-mighty1994fallen}
Gans, Joshua S., and George B. Shepherd. 1994. {``How Are the Mighty
Fallen: Rejected Classic Articles by Leading Economists.''}
\emph{Journal of Economic Perspectives} 8 (1): 165--79.
\url{https://doi.org/10.1257/jep.8.1.165}.

\bibitem[\citeproctext]{ref-lucas1976econometric}
Lucas, Robert E. 1976. {``Econometric Policy Evaluation: A Critique.''}
\emph{Journal of Monetary Economics} 1 (2): 19--46.

\bibitem[\citeproctext]{ref-parkes2015economic}
Parkes, David C, and Michael P Wellman. 2015. {``Economic Reasoning and
Artificial Intelligence.''} \emph{Science} 349 (6245): 267--72.

\bibitem[\citeproctext]{ref-perez2024testing}
Perez-Cruz, Fernando, and Hyun Song Shin. 2024. {``Testing the Cognitive
Limits of Large Language Models.''} Bank for International Settlements.

\bibitem[\citeproctext]{ref-robbins1932essay}
Robbins, Lionel. 1932. \emph{An Essay on the Nature and Significance of
Economic Science}. Macmillan; Co., Limited.

\bibitem[\citeproctext]{ref-stantcheva2023run}
Stantcheva, Stefanie. 2023. {``How to Run Surveys: A Guide to Creating
Your Own Identifying Variation and Revealing the Invisible.''}
\emph{Annual Review of Economics} 15: 205--34.

\bibitem[\citeproctext]{ref-storks2020recent}
Storks, Shane, Qiaozi Gao, and Joyce Y. Chai. 2020. {``Recent Advances
in Natural Language Inference: A Survey of Benchmarks, Resources, and
Approaches.''} \url{https://arxiv.org/abs/1904.01172}.

\bibitem[\citeproctext]{ref-zellers2019hellaswag}
Zellers, Rowan, Ari Holtzman, Yonatan Bisk, Ali Farhadi, and Yejin Choi.
2019. {``Hellaswag: Can a Machine Really Finish Your Sentence?''}
\emph{arXiv Preprint arXiv:1905.07830}.

\end{CSLReferences}



\end{document}
