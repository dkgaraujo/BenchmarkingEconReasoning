% Options for packages loaded elsewhere
\PassOptionsToPackage{unicode}{hyperref}
\PassOptionsToPackage{hyphens}{url}
\PassOptionsToPackage{dvipsnames,svgnames,x11names}{xcolor}
%
\documentclass[
]{article}

\usepackage{amsmath,amssymb}
\usepackage{iftex}
\ifPDFTeX
  \usepackage[T1]{fontenc}
  \usepackage[utf8]{inputenc}
  \usepackage{textcomp} % provide euro and other symbols
\else % if luatex or xetex
  \usepackage{unicode-math}
  \defaultfontfeatures{Scale=MatchLowercase}
  \defaultfontfeatures[\rmfamily]{Ligatures=TeX,Scale=1}
\fi
\usepackage{lmodern}
\ifPDFTeX\else  
    % xetex/luatex font selection
\fi
% Use upquote if available, for straight quotes in verbatim environments
\IfFileExists{upquote.sty}{\usepackage{upquote}}{}
\IfFileExists{microtype.sty}{% use microtype if available
  \usepackage[]{microtype}
  \UseMicrotypeSet[protrusion]{basicmath} % disable protrusion for tt fonts
}{}
\makeatletter
\@ifundefined{KOMAClassName}{% if non-KOMA class
  \IfFileExists{parskip.sty}{%
    \usepackage{parskip}
  }{% else
    \setlength{\parindent}{0pt}
    \setlength{\parskip}{6pt plus 2pt minus 1pt}}
}{% if KOMA class
  \KOMAoptions{parskip=half}}
\makeatother
\usepackage{xcolor}
\setlength{\emergencystretch}{3em} % prevent overfull lines
\setcounter{secnumdepth}{5}
% Make \paragraph and \subparagraph free-standing
\ifx\paragraph\undefined\else
  \let\oldparagraph\paragraph
  \renewcommand{\paragraph}[1]{\oldparagraph{#1}\mbox{}}
\fi
\ifx\subparagraph\undefined\else
  \let\oldsubparagraph\subparagraph
  \renewcommand{\subparagraph}[1]{\oldsubparagraph{#1}\mbox{}}
\fi


\providecommand{\tightlist}{%
  \setlength{\itemsep}{0pt}\setlength{\parskip}{0pt}}\usepackage{longtable,booktabs,array}
\usepackage{calc} % for calculating minipage widths
% Correct order of tables after \paragraph or \subparagraph
\usepackage{etoolbox}
\makeatletter
\patchcmd\longtable{\par}{\if@noskipsec\mbox{}\fi\par}{}{}
\makeatother
% Allow footnotes in longtable head/foot
\IfFileExists{footnotehyper.sty}{\usepackage{footnotehyper}}{\usepackage{footnote}}
\makesavenoteenv{longtable}
\usepackage{graphicx}
\makeatletter
\def\maxwidth{\ifdim\Gin@nat@width>\linewidth\linewidth\else\Gin@nat@width\fi}
\def\maxheight{\ifdim\Gin@nat@height>\textheight\textheight\else\Gin@nat@height\fi}
\makeatother
% Scale images if necessary, so that they will not overflow the page
% margins by default, and it is still possible to overwrite the defaults
% using explicit options in \includegraphics[width, height, ...]{}
\setkeys{Gin}{width=\maxwidth,height=\maxheight,keepaspectratio}
% Set default figure placement to htbp
\makeatletter
\def\fps@figure{htbp}
\makeatother
% definitions for citeproc citations
\NewDocumentCommand\citeproctext{}{}
\NewDocumentCommand\citeproc{mm}{%
  \begingroup\def\citeproctext{#2}\cite{#1}\endgroup}
\makeatletter
 % allow citations to break across lines
 \let\@cite@ofmt\@firstofone
 % avoid brackets around text for \cite:
 \def\@biblabel#1{}
 \def\@cite#1#2{{#1\if@tempswa , #2\fi}}
\makeatother
\newlength{\cslhangindent}
\setlength{\cslhangindent}{1.5em}
\newlength{\csllabelwidth}
\setlength{\csllabelwidth}{3em}
\newenvironment{CSLReferences}[2] % #1 hanging-indent, #2 entry-spacing
 {\begin{list}{}{%
  \setlength{\itemindent}{0pt}
  \setlength{\leftmargin}{0pt}
  \setlength{\parsep}{0pt}
  % turn on hanging indent if param 1 is 1
  \ifodd #1
   \setlength{\leftmargin}{\cslhangindent}
   \setlength{\itemindent}{-1\cslhangindent}
  \fi
  % set entry spacing
  \setlength{\itemsep}{#2\baselineskip}}}
 {\end{list}}
\usepackage{calc}
\newcommand{\CSLBlock}[1]{\hfill\break\parbox[t]{\linewidth}{\strut\ignorespaces#1\strut}}
\newcommand{\CSLLeftMargin}[1]{\parbox[t]{\csllabelwidth}{\strut#1\strut}}
\newcommand{\CSLRightInline}[1]{\parbox[t]{\linewidth - \csllabelwidth}{\strut#1\strut}}
\newcommand{\CSLIndent}[1]{\hspace{\cslhangindent}#1}

\usepackage{algpseudocode}
\usepackage{algorithm}
\usepackage[noblocks]
{authblk}
\renewcommand*{\Authsep}{, }
\renewcommand*{\Authand}{, }
\renewcommand*{\Authands}{, }
\renewcommand\Affilfont{\small}
\DeclareMathOperator*{\argmin}{arg\,min}
\makeatletter
\@ifpackageloaded{caption}{}{\usepackage{caption}}
\AtBeginDocument{%
\ifdefined\contentsname
  \renewcommand*\contentsname{Table of contents}
\else
  \newcommand\contentsname{Table of contents}
\fi
\ifdefined\listfigurename
  \renewcommand*\listfigurename{List of Figures}
\else
  \newcommand\listfigurename{List of Figures}
\fi
\ifdefined\listtablename
  \renewcommand*\listtablename{List of Tables}
\else
  \newcommand\listtablename{List of Tables}
\fi
\ifdefined\figurename
  \renewcommand*\figurename{Figure}
\else
  \newcommand\figurename{Figure}
\fi
\ifdefined\tablename
  \renewcommand*\tablename{Table}
\else
  \newcommand\tablename{Table}
\fi
}
\@ifpackageloaded{float}{}{\usepackage{float}}
\floatstyle{ruled}
\@ifundefined{c@chapter}{\newfloat{codelisting}{h}{lop}}{\newfloat{codelisting}{h}{lop}[chapter]}
\floatname{codelisting}{Listing}
\newcommand*\listoflistings{\listof{codelisting}{List of Listings}}
\usepackage{amsthm}
\theoremstyle{plain}
\newtheorem{theorem}{Theorem}[section]
\theoremstyle{definition}
\newtheorem{definition}{Definition}[section]
\theoremstyle{remark}
\AtBeginDocument{\renewcommand*{\proofname}{Proof}}
\newtheorem*{remark}{Remark}
\newtheorem*{solution}{Solution}
\newtheorem{refremark}{Remark}[section]
\newtheorem{refsolution}{Solution}[section]
\makeatother
\makeatletter
\makeatother
\makeatletter
\@ifpackageloaded{caption}{}{\usepackage{caption}}
\@ifpackageloaded{subcaption}{}{\usepackage{subcaption}}
\makeatother
\ifLuaTeX
  \usepackage{selnolig}  % disable illegal ligatures
\fi
\usepackage{bookmark}

\IfFileExists{xurl.sty}{\usepackage{xurl}}{} % add URL line breaks if available
\urlstyle{same} % disable monospaced font for URLs
\hypersetup{
  pdftitle={Benchmarking economic reasoning in artificial intelligence models (Preliminary: estimation in progress)},
  pdfauthor={Douglas K. G. Araujo},
  colorlinks=true,
  linkcolor={blue},
  filecolor={Maroon},
  citecolor={Blue},
  urlcolor={Blue},
  pdfcreator={LaTeX via pandoc}}

\title{Benchmarking economic reasoning in artificial intelligence models
(Preliminary: estimation in progress)\thanks{This work represents my
opinion and not necessarily that of the BIS. Note: estimation of the
benchmark results in progress; results will be included in the first
complete draft.}}


  \author{Douglas K. G. Araujo}
            \affil{%
                  Bank for International
Settlements, douglas.araujo@bis.org
              }
      
\date{}
\begin{document}
\maketitle
\begin{abstract}
A structural model of economic reasoning combines information filtering,
knowledge association and logic attribution to correctly answer prompts.
The model is adapted for economics but can also be adapted to other
sciences. The reasoning model, along with insights from social
economics, informs the design of an economic reasoning benchmark that
(a) evolves over time in a non-trainable way; (b) measures reasoning
rather than accuracy (ie, disregarding luck and stochastic parroting)
and (c) breaks down reasoning results into its constituent components.
An accompanying training dataset will be publicly available at
publication time, and interested users can submit task proposals. NB:
Empirical estimations are being run. Keywords: Economic reasoning,
benchmark, large language models, artificial intelligence. JEL codes:
C45, C69, C88, C59
\end{abstract}

\begin{quote}
``Machines will be capable, within twenty years, of doing any work a man
can do'' - Herbert Simon, AI pioneer, in 1965
\end{quote}

\section{Introduction}\label{introduction}

Large language models (LLMs), in particular those classified as
generative artificial intelligence (gen AI), increasingly support use
cases in finance and economics (Korinek (2023)), including in central
banks (Araujo et al. (2024)). These models are tested for their ability
to reason, often boasting seemingly incredible results: for example,
OpenAI's GPT-4 boats human-like performance across tests of reasoning
and more than 80\% correct results in academic and professional micro-
and macroeconomics tests (Achiam et al. (2023)). Still, even such
advanced models can fail miserably when it comes to reasoning: for
example, an advanced model can correctly solve a logical puzzle
requiring reasoning about higher order knowledge, only to fail when
irrelevant details are changed (Perez-Cruz and Shin (2024)). Building on
results such as this, this work discusses how to systematically measure
\emph{economic} reasoning, combining literatures on economic thought and
on computer science about gen AI benchmarking. In practical terms, the
task at hand is to come up with a benchmark task for economic reasoning,
a testing mechanism to measures in a comparable way the level of
economic reasoning of an artificial intelligence (AI) model. This task
is of first-order importance given the break-neck speed of evolution of
LLMs (Yang et al. (2023)) and their potential risks (Danielsson, Macrae,
and Uthemann (2022)).

Such benchmark tasks are crucial for the comparison of the abilities of
AI models over time and in the cross-section. Results of benchmark tasks
are now a staple of the evaluation of LLMs by developers when releasing
a model, highlighting its evolution compared to previous versions and to
the main peers. Third-party organisations also compile leaderboards with
running results that allow the general public to keep track of the most
performant models.\footnote{Commonly followed leaderboards include
  LMSYS's ChatbotArena and Huggingface's Open LLM Leaderboard.}
Benchmark tasks are useful because that they provide a comparable metric
on which to track the state-of-the-art for the particular abilities that
each task measures. Usually, this metric is the percentage of correct
answers. However, even if specific tasks have evolved over time to
become more challenging, a key challenge is to separate correct answers
due to probabilistic association or ``stochastic parroting'' (Bender et
al. (2021)) from those that are the result of a reasoning process.

This paper proposes a working model of reasoning that can underlie an
empirical benchmark task: when confronted with a prompt
\(Q\),\footnote{Usually this will be a string of text, but more recent
  models can take multi-model content, ie a combination of text, images,
  videos, sounds, etc.} an AI is said to \emph{reason} correctly if it
responds with an answer \(\alpha\) that simultaneously (a) interprets
the prompt, identifying the relevant information for the task and
filtering away everything else by ignoring or abstract away irrelevant
details, (b) associates \(Q\) with any relevant existing commonsense
knowledge \(\theta\) to answer the question, and (c) applies logical
relations such as deduction and induction to \(Q\) and \(\theta\) to
arrive at the correct answer. Formally, each answer is defined as a
non-parametric function of the following steps: information filtering
\(\phi = f(Q)\), knowledge curation \(\kappa = k(\phi, \theta)\) and
logic attribution \(\lambda = l(\kappa, Q)\),
\(\alpha = A(\lambda, Q)\), where \(A\) is the function that returns the
correct answer from the prompt \(Q\) given a successful implementation
of the component steps. Each of those three steps above are sequential,
and depend on the successful completion of the previous one. The goal is
for this model to be simple and intuitive. Throughout the paper, I
assume AIs respond to their best ability, meaning that they would reason
instead of probabilistically choosing an answer
\(\tilde{a} = \arg \max_{a} L(a | Q)\).

The empirical version of this prompt-answering model is

\begin{equation}\phantomsection\label{eq-empiric}{\hat{\alpha}(q) = \hat{\phi} \, \hat{\kappa}|(\phi=\hat{\phi}) \, \hat{\lambda}|(\phi=\hat{\phi}, \kappa=\hat{\kappa}),
}\end{equation}

where \(\phi = \mathbf{1} \prod_{i}^{M_f} f(q + \epsilon_i^f)\) and
similarly for \(\kappa\) with \(k\) and \(\lambda\) with \(l\), \(M_f\),
\(M_k\) and \(M_l\) are the number of variations \(\epsilon\) introduced
in the seed question \(q\) that seek to identify the model's
interpretation, knowledge association and logic attribution
respectively; the hat denomination points to empirically estimated
versions. This model is estimated by assessing answers from the same AI
model to multiple versions of seed questions \(q \in \mathbf{Q}\), and
\(\hat{\alpha}\) is only considered to be correct when all of the
relevant variations for the same question are answered correctly - in
other words, the AI model has evaded ``banana skins'' that try to trick
it into revealing lack of information filtering, spurious knowledge
association or faulty logics. The key idea is to leverage insights from
the social economics literature and create identifying variation in the
questions \(q\) presented to AI models, adapted from how this is done
with human subjects (Stantcheva (2023)). The benchmark result is then
\(R = M_{\mathbf{Q}}^{-1}\sum_{q \in \mathbf{Q}} \hat{\alpha}(q)\),
where \(M_{\mathbf{Q}}\) is the number of different seed questions. By a
similar token, each of the three steps can be measured separately,
building on their empirical identifications
\(R_{j \in (f, k, l)} = M_{j}^{-1}\sum_{q \in \mathbf{Q}} \hat{j}(q)\).
Note that identification of \(\kappa\) and \(\lambda\) need the
sequential conditionality on the previous steps \(\phi\) and
\(\phi, \kappa\) respectively in order to be identified.\footnote{The
  benchmark result could also be compiled as a metric that is not
  subject to the false ``emerging abilities'' results (Schaeffer,
  Miranda, and Koyejo (2024)), for example the Brier score (Brier
  (1950)).}

This model can be used as a general abstraction for an AI reasoning
ability, but two practical adaptations in \(Q\) can make it a model for
economic reasoning more specifically. First, the scope of topics that
are included in \(Q\) should ideally focus on issues of significance to
economics. At its most essential form, testing for economic reasoning is
the same as probing if the model is able to think in terms of logical
operators on information that is of relevance to economics. However,
this is subjective because economic thought constantly evolves. At the
same time, including only ``classical'' economics carries a high risk of
using material that is containd in the training set of AI models.
Second, even for each given topic, the types of questions considered
relevant in economics are specific. Both of these issues are dealt with
in practice by using recent published academic work as source material
to construct seed questions.\footnote{Similar adaptations could be
  pursued for other fields.} These sources contain content whose topic
and research questions are by definition of interest to the economics
field, and moreover have the advantage of being novel by design,
creating a natural check for the ability of AI models to generalise
reasoning.

This benchmark task is, by design, more difficult than others due to its
sequential conditioning. Is it really necessary to have such a
hard-to-score task? While evidence amounts that large language models
(LLMs) cannot perform advanced reasoning, at least not when fine-tuned
or with access to external reference material or sophisticated math
funcionalities, it is important to have a challenging reasoning
benchmark because the latter two possibilities are increasingly being
used. Both fine-tuning on specific data and plugging LLMs into sources
of knowlege (as in retrieval-augmented generations, or RAGs) or with
plugins as Wolfram Alpha or Mathematica might provide models with some
reasoning ability. For this reason, it is important to have a robust way
of measuring the reasoning abilities of AI models. Another argument is
that part of AI models' reasoning performance out of the box is due to
its limitation to train only on available written language only
(Browning and LeCun (2022)). Of course, this dataset is a very limited
snapshot of human experience, even with massive data compilations.
However, current and future models will be able to leverage a
significant part of the other non-text data (eg, video, audio, pictures)
and thus could reasonably attempt to achieve better reasoning. This
provides another reason to maintain a high bar in reasoning benchmarks.

This benchmark evaluation also addresses a poignant issue for the
economics profession: the lack of publicly available data about how
these benchmarks are created and any, and tested. For example, Achiam et
al. (2023) are not clear about the academic and profession tests on
micro- and macroeconomics that are used amongst various other tests to
measure GPT-4's performance in those fields. Conversely, various other
benchmark tasks do have a publicly available methodology and even
evaluation interface, which greatly facilitates the engagement with
model developers, general users and third-party model evaluators.

A major inspiration in the design of the questions and how they can
generate identifying variations is the social economics literature. A
key reference is Stantcheva (2023). The idea here is that the design of
the questionnaire itself can elicit responses that allow for insight
into non-observable traits such as reasoning. Many of the insights of
this literature carry over naturally to the machine space.\footnote{Actually
  testing whether LMs \emph{do not} parrot or ``organically'' exhibit
  biases or other behaviours that are assumed to be exclusively human
  would be an interesting line of research.}

\subsection{Literature}\label{literature}

This work builds on, and seeks to expand, three general literature
streams. More technical aspects of this work are based on specific
literatures that are discussed within each section.

The first body of works is on benchmarking tasks for AI models. A
substantial body of work creates and discusses model benchmarking in
general; a voluminous and well-organised compilation of references is
Storks, Gao, and Chai (2020). Interested readers are strongly encouraged
to that paper for space considerations, while selected benchmarks are
described in Section~\ref{sec-bench}.

Secondly, this paper draws from insighs in the more general AI reasoning
literature. As other parts of AI development, it is informed and
inspired by neuroscience as well (Hassabis et al. (2017)). An early and
influential contribution is the Chinese room experiment by Searle (1980)
and its resulting arguments that AI could not reason by itself. The
influential works of Bubeck et al. (2023) and Wei, Tay, et al. (2022)
hint at acquisition of advanced capabilities by large-scale language
models such as GPT-4, although Bubeck et al. (2023) also point to many
instances where reasoning breaks. Schaeffer, Miranda, and Koyejo (2024)
present evidence that these ``emerging abilities'' that come with scale
are actually a spurious by-product of the choice of metrics to emasure
these abilities. Mitchell and Krakauer (2023) summarises the
disagreement in the AI academic and practitioner fields as to whether AI
models have some form of understanding, and by implication, potentially
also reasoning. Wei, Wang, et al. (2022) claim that writing the prompt
in a way that offers chain-of-thought examples improves the reasoning
abilities of LLMs, although this was later demonstrated to be as
generalisable (Dziri et al. (2024), Prystawski, Li, and Goodman (2024)).
Browning and LeCun (2022) argue that AI models trained on written
language alone will never be able to reason.

A nascent literature on the evaluation of language models in economic
settings. An early foray into questions related to AI's ability to
conduct economic reasoning is due to Parkes and Wellman (2015). But
their angle is more on how AIs can be used to estimate synthetic
economic agents - machina oeconomicus - ideal versions of purely
rational agents, rather than on the measurement and the implications of
AIs acquiring economic reasoning abilities. In any case, Parkes and
Wellman (2015) see economic reasoning as the ability to understand and
solve complex game-theoretical environments (eg, the poker example). Mei
et al. (2024) do an extensive comparison of personality traits from the
behaviour of ChatGPT with human behaviour in games that require
cooperation, finding that its performance is consistent with humans, and
when it deviates the AI models tend to behave in the altruistic and
cooperative than the mass distribution of humans. Interestingly, ChatGPT
responds differently to different formulations of the same situation. In
contrast to Mei et al. (2024), this paper and its empirical counterpart
are more generall, and discuss reasoning as a whole. Another contrast to
that paper is that the current benchmark is focused on reasoning ability
only, not personality. Perez-Cruz and Shin (2024) illustrate the
brittleness of a leading AI's reasoning, which has markedly lower
performance when trivial details in the prompts are different.
Similarly, Korinek (2023) report (in his Chat 23) that results from a
technical prompt in economics are reasonable but also brittle, with
answers changing when prompt wording changes or even simply if the tasks
are re-ordered.

\subsection{The challenges with reasoning: a simple
illustration}\label{the-challenges-with-reasoning-a-simple-illustration}

This section builds on the intuition that in true reasoning, the result
should be robust to minute perturbations, ie the model is a constant
function over the domain of the input. Formally, both
\(\mathbf{V}(X) = y\) and \(\mathbf{V}(X + \epsilon) = y\) for an
infinitesimal \(\epsilon\), even as \(\mathbf{V}(X') = y', y \neq y'\),
ie it should be a locally constant function. This implies the derivative
with respect to the input prompt is zero. Using as an approachable
example the simplest possible neural network, the logistic regression
\(\mathbf{N}(x) = \sigma(Wx + b)\), such robustness further implies that
\(\frac{d\mathbf{N}}{d x} = \sigma(Wx + b)(1-\sigma(Wx + b))W = 0\).
Because \(W\) cannot be a zero vector in a functioning network that is
responsive to its inputs and \(\sigma(Wx + b)(1-\sigma(Wx + b)) = 0\)
has no solution because neither term is 0 or 1 in a sigmoid function
with finite inputs, the neural network cannot be a constant function.
This extremely simplified example, which holds for recursive
architectures of similarly simple layers, does not bode well for the
robustness of results given small perturbations in the input prompt.

\section{Existing benchmarks}\label{sec-bench}

There are numerous benchmark tasks, and their number is increasing with
the sophistication of newer AI models (Storks, Gao, and Chai (2020)). In
fact, there is nowadays a whole ``benchmark safari'' (to which this
paper is a contribution). Below I summarise important characteristics of
the main ones in each type, to further illustrate why a new benchmark is
needed and more specifically how one geared towards economics can be
constructed.

\subsection{Reference resolution}\label{reference-resolution}

\begin{itemize}
\tightlist
\item
  Measures ability to identify referents (ie, linguistic mentions)
\item
  Complicated task due to ambiguities, requires intense commonsense
  knowledge
\item
  Needs typically handcrafted data
\item
  Datasets tend to be smaller than other benchmarks
\item
  Examples: Winograd, WinoGrande (Levesque, Davis, and Morgenstern
  (2012))
\item
  Browning and LeCun (2023) see the failure of these challenges to
  present a serious challenge to modern LLMs as a sign that lingusitic
  tests are unlikely to be good proxies for reasoning abilities.
\end{itemize}

\subsection{Question answering}\label{question-answering}

\begin{itemize}
\tightlist
\item
  Mixes language processing and reasoning skills
\item
  Some benchmarks can be solved without deep understanding, only from
  linguistic context
\item
  But many Q\&As require external knowledge
\item
  Highlighted example: ARC (Clark et al. (2018))
\item
  Other examples: MCTest, RACE, NarrativeQA, MCScript, ProPara, MultiRC,
  ARCT, SQuAD 2.0, CoQA, QuAC, and many others
\end{itemize}

\subsection{Textual entailment}\label{textual-entailment}

\begin{itemize}
\tightlist
\item
  Entailment: a directional relationship between a statement and a
  hypothesis where a typical person would infer the hypothesis to be
  true given the statement
\item
  Some benchmarks also test ability to recognise contradiction
\item
  Requires substantial commonsense knowledge
\item
  Highlighted example: SherLlic (Schmitt and Schütze (2019))
\item
  Other examples: RTE, conversational entailment, SICK, SNLI, MultiNLI,
  SCITail, and others
\end{itemize}

\subsection{Intuitive psychology}\label{intuitive-psychology}

\begin{itemize}
\tightlist
\item
  Inference of emotions and intensions conditional on description of
  behaviour
\item
  Requires substantial commonsense knowledge
\item
  Highlighted example: SocialIQA (Sap et al. (2019))

  \begin{itemize}
  \tightlist
  \item
    results of increasing performance in other tests suggest some level
    of commonsense knowledge is learnable through data
  \end{itemize}
\item
  Other examples: Triangle-COPA, ROCStories, Event2Mind
\end{itemize}

\subsection{Plausible inference}\label{plausible-inference}

\begin{itemize}
\tightlist
\item
  Associated with abductive reasoning
\item
  Hypothetical, intermediate certainty or even uncertain conclusions
  based on a limited context
\item
  Requires linguistic, common and commonsense knowledge
\item
  Highlighted examples: SWAG (Zellers et al. (2018)), HellaSWAG (Zellers
  et al. (2019))
\item
  Examples: COPA, CBT, ROCStories, LAMBADA, JOCI, CLOTH, ReCoRD,
  AlphaNLI
\end{itemize}

\subsection{Multiple tasks}\label{multiple-tasks}

\begin{itemize}
\tightlist
\item
  Some of them include some tasks requiring some economics-specific
  knowledge
\item
  Examples: bAbI, Inference Is Everything, DNC, GLUE, SuperGLUE
\item
  More recently, BIG Bench (Srivastava et al. (2022))

  \begin{itemize}
  \tightlist
  \item
    BIG = ``Beyond the Imitation Game''
  \end{itemize}
\end{itemize}

\subsection{Expert tasks}\label{expert-tasks}

\begin{itemize}
\tightlist
\item
  Geared towards field-specific knowledge (eg, law, medicine)
\item
  Publicly available tests
\item
  Usually require substantial common and commonsense knowledge
\item
  No economics-specific benchmark that I know of
\end{itemize}

\section{Limitations of existing
benchmarks}\label{limitations-of-existing-benchmarks}

The result from existing benchmarks is largely, if not completely,
directly related to the number of questions correctly answered. However,
this measures only the model's ability to answer correctly, \emph{not
necessarily} its reasoning capabilities. The latter are part of a latent
state space sitting between the input prompt and the answer. More
concretely, for an input prompt \(X\), which includes a question and any
necessary explicit information, the language model is a function
\(\mathbf{M}\) that maps it to a given response:
\(\mathbf{M} : X \to y\). In order to show that it is done by reasoning,
we need tests (and more specifically, measurements) that convey some
information about the inner workings of this function.

As identified by Srivastava et al. (2022), another limitation is that
many benchmarks are not created from questions that are first created by
experts. CONTINUE\ldots{}

\section{A model of reasoning}\label{a-model-of-reasoning}

The model presented in the Introduction is arguably simplistic, but is
fundamented on considerable body f literature that combines sensory
pathways, knowledge combination and the execution of logical. This
section goes into more detail about the choices in the reasoning model.

\subsection{Information filtering}\label{information-filtering}

Reasoning should be robust to irrelevant input or to changes in minutiae
that are not crucial for the logic relations to follow. This result
follows as a result of the efficient coding hypothesis in physiology
(Barlow et al. (1961), Olshausen and Field (1996), Loh and Bartulovic
(2014)), which postulates that sensory pathways need to reduce the
dimensionality of inputs in living organisms. This insight is itself a
biological and social observation ispired by Shannon's (1948) landmark
information theory. The AI literature has of course known this of years,
and it inspired the concept of attention (Larochelle and Hinton (2010),
Mnih et al. (2014)), which later inspired the self-attention and
ultimately the game-changing transformer architecture (Vaswani et al.
(2023)).

The first step, filtering the received impulses (ie, the prompts),
involve correctly judging what is relevant and what is not relevant.
This is similar for example to how the brain receives an incredible
amount of sensory inputs but chooses to focus only on those that are
more relevant instead of being overwhelmed with everything else, an
observation that has inspired dimensionality-reduction algorithms (eg,
isometric mapping, or IsoMap, by Tenenbaum, Silva, and Langford (2000)
describes how to find global optima while also defining the (much lower)
degrees of freedom in a high-dimensional input).

Appropriate perception should understand that the information of
relevance to understanding a problem is actually much lower-dimensional.
Beyond the biological (and more specifically neural) requirement for
inputs to focus on the most important aspects, this observation is also
consistent with a bedrock of the machine learning literature, the
manifold hypothesis. This hypothesis is based on the theoretical and
empirical observations that most real-world realisations are
high-dimensional embeddings of much lower-dimensional
manifolds.\footnote{For example, Pope et al. (2021) study the intrinsic
  underlying dimensionality of the manifold of image datasets and find
  them to be significantly lower than their observed dimensionality. In
  practice, inputs can even be said to be \emph{union of manifolds} (as
  verified by Brown et al. (2022) with image datasets in an exercise
  similar to the one by Pope et al. (2021)), which means that each
  manifold has its own intrinsic dimensionality that is not forced upon
  the other manifolds. This perspective affords flexibility in the
  interpretation of identifying variations because they don't
  necessarily need to probe the same dimensions at each task.} Cayton
(2008) offers an early review of the main algorithms for estimating
empirically the underlying manifold. Bengio, Courville, and Vincent
(2013) discusses extensively the idea of representation learning (and
its various techniques, mostly unsupervised), which can be seen as
manifold learning. However, they might also approximate the wrong
manifold or not have a single solution to a same manifold (Lee (2023)).

The requirement for perception modulation is also aligned with
Prystawski, Li, and Goodman (2024)'s finding that reasoning abilities in
LLMs require ``locality'' in concepts until a final link between a
prompt and its final answer (if far away) can be achieved. In a way,
this is also similar to the small world network model (Kleinberg
(2000)). In humans, the literatures on rational inattention and
neuroeconomics (Sims (2003), Caplin, Dean, and Leahy (2022), Dean and
Neligh (2023), Hébert and Woodford (2021)) models human information
processing as subject to a cost that grows with the informational
content, which is a closer representation of how people actually process
information. In linguistics, the information bottleneck literature
discusses how ideas are compressed into words by a trade-off between
lexicon complexity vs accuracy (Zaslavsky et al. (2018)), and more
recently also consistency (Chen, Futrell, and Mahowald (2023)).

\subsection{Knowledge association}\label{knowledge-association}

Assuming a prompt has been correctly parsed, the reasoning mechanism
must now match it with the relevant knowledge, which can come from the
prompt itself or from commonsense knowledge.

Knowledge can be linguistic, common or commonsense (Davis and Marcus
(2015)). Mahowald et al. (2023) uses insights including from
neuroscience to distinguish the first type of knowledge with the latter
two (grouped as ``functional'' knowledge), and argue that LLMs have
essentially mastered the former while still having a spotty record on
the latter. For example, LLMs learn grammar, semantic, hierarchical
structures, abstractions and constructions that provide a realistic
linguistic knowledge.

Bransford and Johnson (1972) show in experiments that contextual
knowledge (in this case akin to commonsense) are essential for proper
understanding in humans. The first experiments tested understanding by
subjects of a grammatically correct, non-metaphorical passage that
required an unusual and very specific, but highly relatable image as
context for proper understanding. Note that these characteristics of the
passage (correct, non-metaphorical) and of the context (unusual but
relatable and easy to understand) both contribute to isolate the
identification of this exercise in the aspect of whether contextual
knowledge is required. \footnote{Interestingly, the same paper also
  demonstrates that prior knowledge itself is not necessarily readily
  available but needs to be ``activated''. This is not further discussed
  in the context of this paper as it is not a mechanism necessary for
  measuring reasoning abilities in AI models.}

Which type of knowledge to match to the prompt? One way of seeing this
is through the lends of a query in a knowledge graph. Kleinberg (1999)
distinguishes specific and broad queries, each giving rise to one
problem: that of a scarcity of correct answers and abundance of correct
answers, respectively. Further, Kleinberg (1999) offers the fundamental
ideas of \emph{authority}. Similarly, Kleinberg (1999) acknowledges that
measuring the authority level of a node in a knowledge graph from
explicit information alone (what he calls \emph{endogenous} measure). On
the contrary, even so much as using strings from the query itself might
mislead answers due to an abundance of other sources that are based on
the string and a scarcity of correctly authoritative sources that use
the string. Interestingly, while the principal eigenvector of the square
of the adjacency matrix offers the weights of authoritativeness
especially for broad queries, the non-principal eigenvectors can offer
insights into the authoritativeness of more specialised queries, and
also due to their negative entries offer authorities of different
perspectives (ie, weighting pros and cons).

So the lower-rank approximation of the knowledge graph should vary with
how broad/specific a query is, and also with the level of pros/cons
required. In Kleinberg (1999), the authoritativeness measures comes from
the eigenvectors of \(A^TA\), with the principal eigenvector being the
used as a broad authoritativeness metric, and the \(n\)th eigenvectors
for \(n>1\) as the more specific, and potentially discordand,
authorities. This implies that having question variations that refer to
opposite views in the prompt should also help probe the level of
knowledge (given the theoretical model that it is a function of existing
knowledge graph) versus response flipping due to probabilistic
association with the terms introduced to describe the pros and cons.

\subsection{Logic attribution}\label{logic-attribution}

Once a prompt has been correctly parsed to focus only on relevant
information and the necessary knowledge has been associated with it, an
AI model can assign specific logic relationships between different
components of a prompt and the existing knowledge. Commonly studied
logic relationships are induction, deduction, analogy, abstraction,
abduction(Walton (2014), Johnson-Laird, Khemlani, and Goodwin (2015),
Davis and Marcus (2015), Dziri et al. (2024)), among others. Naturally,
an essential condition for appropriate use of logical relationships is
that the inputs \(\phi\) and \(\kappa\) are correct; otherwise, even the
purest application of logic would either lead to a failure or to a
correct answer by chance.

\begin{definition}[Logic
relationship]\protect\hypertarget{def-logic}{}\label{def-logic}

A logic relationship is a \ldots{}

\end{definition}

Definition~\ref{def-logic} introduces the idea of logic relationships.
One example\ldots{}

Similar to many other social disciplines, economics requires the
analytical judgment referred to by Robbins (1932) in the analyses of
events as a basis to extrapolate and predict, and this has a bearing on
how economic reasoning should be benchmarked. Economic inference depends
primarily on articulating unobservable quantities, theorecised and
estimated on the basis of observable measures. This is unlike other
major disciplines. For example, in human and veterinary medicine, all
physiological and pathological variables of clinical importance are
observable, even if that is not yet technologically feasible today. In
the medical sciences, theoretical models merely fill in the gaps in the
absence of a technologically feasible complete measurement. In contrast,
many economically relevant quantities are latent variables that cannot
by definition be observed, and always require a model applied to data to
be estimated, implicit or not.

A quantitative test for economic reasoning must take into account that
selecting a correct answer in an economics question through reasoning
will always depend on an unobserved transformation of the information
received and the existing knowledge. AI models may also happen to choose
the correct answer from either luck of through simple token probability.
It is easy to see why a correct answer selected by chance is not
informative about the reasoning abilities of a model. Since reasoning
should be robust to perturbations in the input data, is locally
complete, meaning that an LM that can correctly deduce that A implies B
is also able to understand that A' does not imply B, or that A does not
imply B'. In other words, a logic relationship that appears to be
correct but whose obvious corolary is not achieved by an LLM cannot be
said to have been reasoned in the first place.

\subsection{Reasoning iself as a low dimensional
operation}\label{reasoning-iself-as-a-low-dimensional-operation}

Why is the information filtering important for identifying reasoning?
Since proper reasoning needs to be insensitive to unimportant details,
and the vector of changes depends on logical relationships between
components, the set of all ``reasonable'' constructions is not obtained
at random but reflects a lower-dimensional, underlying space.

Gilboa et al. (2014) argues that economic reasoning works by creating
positively wrong but conceptually useful representations of reality,
even when economics is studying particular cases. A marked
characteristic of such models is their preference for simplicity, a
theme also explored by Gilboa and Schmeidler (2010), who study the
matching of economic theories to empirical data, generalising the
evaluation of how reasonable a theory is through a combination of their
likelihood (or goodness-of-fit) with a penalising factor for their
complexity. Intuitively, this simplicity in reasoning is suggestive of
the manifold hypothesis in reasoning as well. Gilboa, Minardi, and Wang
(2023) sees rationality, or reasoning, also as a robustness to trivial
detail.

A related, more empirically applicable perspective, is the possibility
to identify models partially using random sets, ie abstracting away from
point identification to situations where the data is incomplete or is
described as an interval (Beresteanu, Molchanov, and Molinari (2012)).
In other words, combining assumptions with available data to inform the
estimation (even if partial) of a parameter of interest (Molinari
(2020)), which are situations that could arise for example when the data
originates from a lower-dimensional manifold but is observed as an
embeddeding in higher dimensions.

All of these insights above, together with the points made in each
specific step, lend credence to the intuition that the reasoning process
itself entails some dimensionality reduction, in line with the existence
of evidence that some real-world data are in fact realisations of a
manifold (Pope et al. (2021)).

\subsection{Identification}\label{identification}

Due to the sequential nature of Equation~\ref{eq-empiric} and the
desirability of measuring an AI model's performance across the three
reasoning steps, identification of the model obtains from a combination
of analysis to find how the values can be estimated empirically to be
one or zero. This then leads to the creation of identifying variation
during the question generating process as described in
Section~\ref{sec-variation}.

For each question \(q\), \(\phi\) only depends on the specific prompt at
hand (ie, \(q + \epsilon_f\)) and is independent of the two other
values. When estimated from the question data it is also independent of
the AI model's knowledge set \(\theta\). Once an estimate for
\(\hat{\phi}\) is available, it can be used to condition \(\kappa\)
along with the AI's fixed knowledge set, which does not change across
observations of \(q\) or \(q + \epsilon_k\), and therefore is absorbed
away since the estimation is done for each AI. Finally, \(\hat{\phi}\)
and \(\hat{\kappa}\) are used to condition the estimates of \(\lambda\)
based on \(q + \epsilon_l\).

\begin{theorem}[Identification of reasoning
abilities]\protect\hypertarget{thm-identification}{}\label{thm-identification}

If\ldots{}

\end{theorem}

\begin{proof}
By induction.
\end{proof}

\section{Reasoning about economics}\label{reasoning-about-economics}

The model above allows us to estimate reasoning while also breaking down
some of its components to better understand them. For example, we can
estimate any errors in reasoning into an issue with information
filtering, knowledge association and logic attribution. The goal of this
section is to clarify that the working definition of economic reasoning
used in this work is very broad and is not meant to assign one or
another form of reasoning with greater weight, let alone one economic
school of thought over the other. Rather, the goal is to reflect over
time also slower-moving features of our profession, such as the ebb and
flow of schools of thought (Pribram (1953)).

Gilboa et al. (2022) distinguish between three types of inquity in
economic theory: economics itself (analysis of economic phenomena),
development of economic methods (the development of analytical tools
needed to study economic phenomena) and the methodology of economics
(the research/scientific endeavour in economics, including but not
limited to theory).\footnote{In fact, Gilboa et al. (2022) even allude
  to the blurred lines between economics and the philosophy or sociology
  of economics. I don't go ino these differences here.} Naturally, they
would all be in scope of this exercise.

Another insight into \emph{economic} thinking is from the thought
experiments first introduced by Marshall (1890) - the ceteris paribus
idea - and then later Ragnar Frisch and Trygve Haavelmo, more recently
elaborated in more detail and more generally by J. Heckman and Pinto
(2015), including the important distinction between correlation and
causation. Marschak and Andrews (1944) start their influential paper by
acknowledging that economists can't conduct experiments (although that
has been relaxed somewhat, it still remains the case at least in
macroeconomics). Reasoning in economics can be taken as exploring the
latest space through models, or thought experiments, an insight
generalised by J. J. Heckman and Pinto (2023). Pribram (1953) adds to
argue that even the areas of focus - the answer to \emph{what matters} -
changes over time.

Other sciences, in contrast, might have more concrete definitions of
reasoning. For example, the human and veterinary medicines require hard
data to conduct reasoning. In its absence, due for example to
technological, biological or economic constraints, professionals need to
theorise and especially rely on abductive reasoning. But if one day all
of these constraints were relaxes and medical and veterinary doctors had
access to all possible data about their patients, they would only need
to reason with respect to the physiological or pathological relationship
between these observed variables. Economists, in contrast, will always
need to conduct thought experiments and think in latent terms to conduct
any type of meaningful economic reasoning.

\section{Empirical estimation}\label{sec-variation}

Recall from Equation~\ref{eq-empiric} that estimation requires an
evaluation of several variations of the same seed question \(q\) for
each component. These variations form the empirical data. Questions vary
with respect to their adversarial aspect; it is this variation within
each question that allows the empirical estimation of the effects
associated with interpretation or with knowledge. Most of the variations
are originally those tested in Alzahrani et al. (2024). The variation in
response between the questions within each task will comprise the
evaluation of the actual reasoning capabilities. As alluded to before,
the variations are organised into those that measure the stability of a
response to adversarial interpretation answers, and those that measure
the stability across the knowledge dimension. In practice, each task has
hundreds of different \(q\). These groups are described in more detail
next.

\subsection{Adjusted adversarial
filtering}\label{adjusted-adversarial-filtering}

The key idea in this section is to use the adversarial filtering process
proposed by Zellers et al. (2018) and Zellers et al. (2019), adapted to
the current application. A generative model creates an initial set of
incorrect responses to each video. This is then fed to the adversarial
filtering routine, which is executed iteratively. First, the source data
is split into training and testing sub-samples. An AI model is trained
on the training sub-sample and used to identify those in the testing
data that are easier to correctly answer. Those easy alternatives are
then taken out of the sample, and newly generated alternatives replace
them. The process is repeated until stabilisation.

A few adjustments are needed for the current case: this process occurrs
separately for each of the seed questions \(q\) and each of the three
steps. This results in a collection of
\(\mathbf{W}_q^{(1)} = \epsilon^f, \epsilon^k, \epsilon^l\) for each
\(q\). The iterative filtering then proceeds as described above, until
the performance of adversarial filters have degraded to an arbitrarily
low point.

\$

\begin{algorithm}
  \caption{Algorithm for finding server indices using OFG}

  \begin{algorithmic}
    \Statex \Comment { \%comment: servers[] contains the index of servers whose         data rate are sorted in descending order\%}
    \State servers[]= index(of all servers) 
    \State serverIndex[]=servers[0..K]
    \State linearlyIndependentServerIndex[]=0
    \State $[Z] \leftarrow 0$
    \For  {$i=0$ to $serverIndex.length$} 
    \Statex\Comment{ \%comment: find the equation corresponding the serverIndex        from the mapping at the File Server\%} 
    \State        $eqn= equation(serverIndex[i])$ 
    \Statex\Comment{ \%comment: try insert equation into Z using OFG\%} 

    \EndFor end for 
    \While{ ( linearlyIndependentServerIndex.length!=K ) } 
    \Statex\Comment{\%comment: remove all the server index which were not inserted in Z\%}  
    \State temp[]=serverIndex[]-linearlyIndependentServerIndex 
    \If{  (linearlyIndependentServerIndex.length=K) }
    \State break
    \EndIf  
    \EndWhile  
  \end{algorithmic}
\end{algorithm}

\$

\subsection{Variations related to information
filtering}\label{variations-related-to-information-filtering}

There are several classes of variations that can help test an AI model'
interpretation of the input, ie which information to focus versus to
ignore.

\textbf{Choice variations}: In this dimension, each variation of a seed
question would have the same set of choices, but in varying order.
Following Alzahrani et al. (2024), this variation can be random,
leverage bias to include the correct answers at the beginning or end of
the set, explore identifying choice alternatives with uncommon answer
choice symbols (eg, non-standard letters instead of a-d), and even
common but unordered answer choice symbols.

\textbf{Word variations}: The main idea here is to introduce or change
words that are irrelevant and not part of the main concept. This is
along the lines of the test conducted by Perez-Cruz and Shin (2024).
Once a human annotates a seed question to identify words that can be
safely changed without changing the underlying reasoning task for
questin \(q\), an adversarial filtering model would create multiple
alternatives from real and when needed simulated words.

\textbf{Irrelevant information}: The same seed question can be augmented
to include irrelevant information to varying degrees, including flooding
the question in the midst of completely random information. This can be
done through random string, strings guaranteed to not be relevant to the
case in point, or even completely gibberish words.

\subsection{Variations related to
knowledge}\label{variations-related-to-knowledge}

Recall that the seed questions use as little jargon as possible. This is
useful for two reasons. First, using non-jargon greatly increases the
chance that all the words in \(q\) are part of the training data
dictionaries, and thus any performance issue is related to the network
architecture itself and not to its choice of dictionary size. Second,
less technical words probably carry more generalistic meaning than
specific words, and thus could trick stochastic parrot models into
providing answers that have a closer probabilistic association with
these general words more than it would if the model purely reasoned
(even if it ultimately reasoned incorrectly).

Adversarially testing knowledge involves include faux technical jargon
in a way that would materially change the answer if wrongly interpreted
as existing jargon. For example, ask if the interpretation of a given
statement would change if ``the heteroscedasticity is over-identified in
the vector space''. Obviously such a passage would not make a sense to
an expert but could fool an AI model. Comparing responses between both
should indicate the level of knowledge used by the model. The location
for the random faux jargon can be specified by humans with a special
token, or completely randomly set by an adversarial filtering model.

A third source of variation to probe a model's knowledge comes from the
intuition that knowing about something also involves knowing how to
oppose ideas about it, analogous to how information authorities in
non-principal eigenvectors are opposed to each other in Kleinberg
(1999). In the benchmark model, this results in a third source of
knowledge variation whenever relevant to the question \(q\) that retains
the same prompt but includes specific wordings related to pros and cons.

\subsection{Variations related to logic
attribution}\label{variations-related-to-logic-attribution}

The primary way to test for logic is to include a term that leads to the
reverse conclusion, and check if that would alter the results. The
implications from this filter to the analysis of logic is obvious. In
practice, this can be implemented by humans but also by adversarial
filtering.

Another way to test logic is to conduct the flip-flop experiment: simply
asking LLMs to confirm their answers often make them switch answers,
even if their original response was correct (Laban et al. (2023), Xie et
al. (2023)). The key idea of these tests is to follow up a response with
a reply that either asks for confirmation (``are you sure?'') or that
doubts the AI models's answer in some way, even if it was originally
corrected. Controlling for the ability to correctly interpret and filter
the incoming information, the response shouldn't change since no other
change has occurred in the AI models' filtering ability or knowledge
base (neither on the original model weights obviously but also including
the information content of the prompt).

\section{Practical considerations}\label{practical-considerations}

\subsection{Avoiding spillover into training
data}\label{avoiding-spillover-into-training-data}

The strategy to use newly published academic papers as sources might
broadly avoid that most of the content has been used in AI model
training. However, most published papers in economics are previously
published as working papers, which means they are potentially in the
public domain at training time so cannot be guaranteed to be completely
novel. While this is mitigated by the arguably low dissemination of
secondary material about working papers (for example, one could
reasonably conjecture that few recent working papers immediately become
the topic of teaching notes or are referred to in more detail by other
papers), a more robust practical strategy is needed, especially as the
training dataset of many of the most advanced models is not publicly
known.

One way of dealing with this is by introducing in a variation of the
questions a random string that is almost guaranteed to be unique and
that is not found in common text datasets used to train LLMs. This is of
course not perfect, because it cannot guarantee that the original paper
is not part of training data, but can at least ensure that if the seed
questions themselves are for some reason used to train models, this
could be identified by model developers (and if the training data is
available, also by third-party evaluators).

\subsection{Lessons from human
surveys}\label{lessons-from-human-surveys}

I use a considerable amount of specific advice on human surveys from
Stantcheva (2023) to generate identifying variation in the questions.
Specifically, all the questions avoid jargon to the best extent
possible, and only include questions that are either of the coeteris
paribus type, or that include as options assessments on the statements
of the form ``correct'', ``incorrect'', ``equal'' or ``I don't
know''.\footnote{Future versions of this benchmark could also include
  open ended questions (as in Ferrario and Stantcheva (2022)), and even
  follow-up questions (``are thre any other reasons''). These open-ended
  questions that are similar in nature to closed-end questions could be
  assessed by a fine-tuned LLM.} Particular care is taken with respect
to introducing variations in the seed questions that can help measure
each of the three reasoning components of information filtering,
knowledge association and logic attribution.

Consideration is given to whether each question should be presented to a
separate instance of the LM, or the full questionnaire could be shared
in the same ``chat'', which would be akin to the ``few-shot'' prompting.
Another practical advice as part of the estimation is to prototype the
questions (I used GPT-4 for the prototyping).

\subsection{Desirable characteristics of a
benchmark}\label{desirable-characteristics-of-a-benchmark}

A benchmark task for economic reasoning should ideally have the
following characteristics in order to be useful and maintain relevance
even in a scenario where model developers are able to acquire a
significant body of economically relevant texts (eg, new papers).

\textbf{Inform performance on different components of reasoning}: An
ideal benchmark can help practitioners intuitively grasp the performance
of the models in each major ``task'' that is performed in the process of
reasoning. This would help developers and users better understand what
the models are good or bad at, and judge their adequateness accordingly.
It can also support a more granular understanding of the acquisition of
reasoning capabilities throughout the training process and scaling of
language models (Biderman et al. (2023)).

\textbf{Evolve over time}: Economic reasoning evolves over time. For
example, the Lucas critique (Lucas (1976)) was influential in shifting
macroeconomic modelling, while the credibility revolution described in
Angrist and Pischke (2008) was similarly influential in microeconomic
work. A historical perspective on the thought about causality going back
to the early 18th century is found in J. Heckman and Pinto (2015), and
Debreu (1984) describes the evolution of economic theory up until that
point. Lewbel (2019) offers a historical perspective on the issue of
identification. For this reason, it is important to consider new works
as they are incorporated in the live economic debate. This can be most
directly done by drawing from academic papers in general interest
economic journals, which benefit from wide impact in the profession.
However, there are two main drawbacks of using academic papers to proxy
for the development of economic reasoning over time. The first is the
widely discussed publication bias (Andrews and Kasy (2019)), but a
perhaps equally important issue is that of unobserved false negatives:
if many contributions that are now considered classics have been
previously rejected (Gans and Shepherd (1994)), there are probably many
others who will not be available for the incorporation as a benchmark
task.

\textbf{Make data available}: Availability of data is crucial for
developers to test their models in-house, and for model evaluators to
suggest improvements to this benchmark. For this reason, an initial set
of publicly available \(Q_{\text{public}}\) containing \(q\)s and their
variations will be put in the public domain. Periodically, as a new set
\(Q_{\text{hidden}}\) is added, older questions will also be made
available, ensuring developers will have access to a rolling set of new
testing material.

\textbf{Cover different levels of economic reasoning}: An ideal economic
reasoning benchmark tests whether the model is able to recognise
increasingly sophisticated levels of economic reasoning. When faced with
\(Q\) that contains a statemet of the economic problem, and a summary of
the methodology and main findings, an AI model must recognise the type
of analysis that was conducted. Drawing from the definitions more
recently stated in J. J. Heckman and Pinto (2023), those are, in order
of analytical prowess, (a) the impact of a given intervention in a
specific environment; (b) understanding the mechanisms by which the
intervention might work; (c) forecasting the effects of the same
intervention in other environments or states of the world; and (d)
forecasting the effects of never-before-implemented interventions in
various environments.

\textbf{Receive inputs from the public}: In order to truly reflect the
breadth and diversity in economic thought, an ideal benchmark should be
open to receiving suggested questions from the public.\footnote{This
  approach was followed, for example, by Srivastava et al. (2022).} For
example, economists publishing a new paper could suggest a source
question based on their work. A practical way to achieve this is to
create clear instructions and a standardised form that would be filled
by that external user presenting the suggestion, coupled with a script
that takes in the source quesiton(s) and introduces the necessary
variations in information, knowledge and logic to achieve
identification. The author or other maintainers of the benchmark will
then review the submissions..

\section{Results}\label{results}

(to be filled once first estimations are concluded)

\section[Preliminary considerations]{\texorpdfstring{Preliminary
considerations\footnote{This section is the basis for the conclusions in
  a future version after the empirical estimation is completed.}}{Preliminary considerations}}\label{preliminary-considerationsconcl}

In this paper I propose a working model of economic reasoning that can
be used to benchmark AI models' reasoning abilities across three
sequential cognitive tasks: filtering the incoming information,
associating it with existing knowledge, and performing logic tasks to
reach a correct answer. The model in this paper resembles the more
sophisticated idea by LeCun (2022) that AIs require a combination of
``mental'' modules that can separably executve perception (eg, take in a
prompt that might be text-only or text-and-image), calculate the cost of
processing, and imagine action sequences.

Understanding AI models' ability to reason and go beyond a pure
probabilistic exercise is crucial as these models have an increasing
importance in society. For example, models that suggest economic actions
to people should better reflect latent, structural models that represent
people's preferences or well-being rather than simply a prediction based
on their observed behaviour, as argued by Kleinberg et al. (n.d.).
Models that reason better can rise up to the challenge of learning
actual metrics of interest instead of real-world measurements, because
the latter have added noise from human cognitive and heuristics
limitations, which are then amplified by multiple types of biases in
data (CITE paper on multiple biases). And with the increasing linguistic
prowess of language models, their use in the economic research process
(Korinek (2023), Ludwig and Mullainathan (2024)) is likely to increase,
putting a premium on the ability to measure how well these models can
reason.

As economic agents and policymakers harness generative artificial
intelligence (AI) to reap considerable efficiencies, and thus their
societal footprint becomes larger, a benchmark for economic reasoning is
needed. I suggest ways to implement such a benchmark, and measure the
current performance of a selected list of LMs. This benchmark is
designed from the beginning to be continuously challenging for AI models
to solve, anticipating further gains in their performance.

Let me conclude with Ken Arrow's impossibility theorem (Arrow (1950)),
or rather the story of how he achieved this incredibly influential
result. Arrow first attempted to improve upon two-century-old
Condorcet's paradox, and studied ways in which individual preferences
could be aggregated while satisfying some intuitive conditions. It was
only through repeated failures to do so that he switched the focus to
attempting to prove its impossibility. While Arrow can be safely used as
a prime example of economic reasoning, the point this anecdote
illustrates is that breakthroughs in economic knowledge require also
inspiration (in this case from the appeal of addressing Condorcet's
paradox) as well as persistence and ability to change one's focus. The
current work focuses on developing robust benchmarks of models'
reasoning abilities in economics; further work exploring their
contributions to inspiration\footnote{Korinek (2023) illustrates use of
  AI models to help economists have new ideas for work, and Ludwig and
  Mullainathan (2024) presents ways machine learning can contibute to
  generation of hypothesis.} and to methodological assistance (as in the
example to change focus) are also warranted for a more complete
assessment of models' abilities to provide cognitive support to human
economists.

\section*{References}\label{references}
\addcontentsline{toc}{section}{References}

\phantomsection\label{refs}
\begin{CSLReferences}{1}{0}
\bibitem[\citeproctext]{ref-achiam2023gpt}
Achiam, Josh, Steven Adler, Sandhini Agarwal, Lama Ahmad, Ilge Akkaya,
Florencia Leoni Aleman, Diogo Almeida, et al. 2023. {``GPT-4 Technical
Report.''} \emph{arXiv Preprint arXiv:2303.08774}.

\bibitem[\citeproctext]{ref-alzahrani2024benchmarks}
Alzahrani, Norah, Hisham Abdullah Alyahya, Yazeed Alnumay, Sultan
Alrashed, Shaykhah Alsubaie, Yusef Almushaykeh, Faisal Mirza, et al.
2024. {``When Benchmarks Are Targets: Revealing the Sensitivity of Large
Language Model Leaderboards.''} \emph{arXiv Preprint arXiv:2402.01781}.

\bibitem[\citeproctext]{ref-andrews2019identification}
Andrews, Isaiah, and Maximilian Kasy. 2019. {``Identification of and
Correction for Publication Bias.''} \emph{American Economic Review} 109
(8): 2766--94.

\bibitem[\citeproctext]{ref-angrist2008mostly}
Angrist, Joshua D., and Jörn-Steffen Pischke. 2008. \emph{Mostly
Harmless Econometrics: An Empiricist's Companion}. Princeton University
Press.

\bibitem[\citeproctext]{ref-araujo2024artificial}
Araujo, Douglas Kiarelly Godoy de, Sebastian Doerr, Leonardo Gambacorta,
and Bruno Tissot. 2024. {``Artificial Intelligence in Central
Banking.''}

\bibitem[\citeproctext]{ref-arrow1950difficulty}
Arrow, Kenneth J. 1950. {``A Difficulty in the Concept of Social
Welfare.''} \emph{Journal of Political Economy} 58 (4): 328--46.

\bibitem[\citeproctext]{ref-barlow1961possible}
Barlow, Horace B et al. 1961. {``Possible Principles Underlying the
Transformation of Sensory Messages.''} \emph{Sensory Communication} 1
(01): 217--33.

\bibitem[\citeproctext]{ref-bender2021dangers}
Bender, Emily M, Timnit Gebru, Angelina McMillan-Major, and Shmargaret
Shmitchell. 2021. {``On the Dangers of Stochastic Parrots: Can Language
Models Be Too Big?🦜.''} In \emph{Proceedings of the 2021 ACM Conference
on Fairness, Accountability, and Transparency}, 610--23.

\bibitem[\citeproctext]{ref-bengio2013representation}
Bengio, Yoshua, Aaron Courville, and Pascal Vincent. 2013.
{``Representation Learning: A Review and New Perspectives.''} \emph{IEEE
Transactions on Pattern Analysis and Machine Intelligence} 35 (8):
1798--828.

\bibitem[\citeproctext]{ref-BERESTEANU201217}
Beresteanu, Arie, Ilya Molchanov, and Francesca Molinari. 2012.
{``Partial Identification Using Random Set Theory.''} \emph{Journal of
Econometrics} 166 (1): 17--32.
https://doi.org/\url{https://doi.org/10.1016/j.jeconom.2011.06.003}.

\bibitem[\citeproctext]{ref-biderman2023pythia}
Biderman, Stella, Hailey Schoelkopf, Quentin Anthony, Herbie Bradley,
Kyle O'Brien, Eric Hallahan, Mohammad Aflah Khan, et al. 2023.
{``Pythia: A Suite for Analyzing Large Language Models Across Training
and Scaling.''} \url{https://arxiv.org/abs/2304.01373}.

\bibitem[\citeproctext]{ref-BRANSFORD1972717}
Bransford, John D., and Marcia K. Johnson. 1972. {``Contextual
Prerequisites for Understanding: Some Investigations of Comprehension
and Recall.''} \emph{Journal of Verbal Learning and Verbal Behavior} 11
(6): 717--26.
https://doi.org/\url{https://doi.org/10.1016/S0022-5371(72)80006-9}.

\bibitem[\citeproctext]{ref-brier1950verification}
Brier, Glenn W. 1950. {``Verification of Forecasts Expressed in Terms of
Probability.''} \emph{Monthly Weather Review} 78 (1): 1--3.

\bibitem[\citeproctext]{ref-brown2022verifying}
Brown, Bradley CA, Anthony L Caterini, Brendan Leigh Ross, Jesse C
Cresswell, and Gabriel Loaiza-Ganem. 2022. {``Verifying the Union of
Manifolds Hypothesis for Image Data.''} In \emph{The Eleventh
International Conference on Learning Representations}.

\bibitem[\citeproctext]{ref-BrowningLeCun2022}
Browning, Jacob, and Yann LeCun. 2022. {``AI and the Limits of
Language.''} \emph{Noema}.

\bibitem[\citeproctext]{ref-BROWNING2023104031}
---------. 2023. {``Language, Common Sense, and the Winograd Schema
Challenge.''} \emph{Artificial Intelligence} 325: 104031.
https://doi.org/\url{https://doi.org/10.1016/j.artint.2023.104031}.

\bibitem[\citeproctext]{ref-bubeck2023sparks}
Bubeck, Sébastien, Varun Chandrasekaran, Ronen Eldan, Johannes Gehrke,
Eric Horvitz, Ece Kamar, Peter Lee, et al. 2023. {``Sparks of Artificial
General Intelligence: Early Experiments with GPT-4.''}
\url{https://arxiv.org/abs/2303.12712}.

\bibitem[\citeproctext]{ref-caplin2022rationally}
Caplin, Andrew, Mark Dean, and John Leahy. 2022. {``Rationally
Inattentive Behavior: Characterizing and Generalizing Shannon
Entropy.''} \emph{Journal of Political Economy} 130 (6): 1676--1715.

\bibitem[\citeproctext]{ref-cayton2008algorithms}
Cayton, Lawrence. 2008. {``Algorithms for Manifold Learning.''}
eScholarship, University of California.

\bibitem[\citeproctext]{ref-chen2023information}
Chen, Sihan, Richard Futrell, and Kyle Mahowald. 2023. {``An
Information-Theoretic Approach to the Typology of Spatial
Demonstratives.''} \emph{Cognition} 240: 105505.

\bibitem[\citeproctext]{ref-clark2018think}
Clark, Peter, Isaac Cowhey, Oren Etzioni, Tushar Khot, Ashish Sabharwal,
Carissa Schoenick, and Oyvind Tafjord. 2018. {``Think You Have Solved
Question Answering? Try Arc, the Ai2 Reasoning Challenge.''} \emph{arXiv
Preprint arXiv:1803.05457}.

\bibitem[\citeproctext]{ref-danielsson2022artificial}
Danielsson, Jon, Robert Macrae, and Andreas Uthemann. 2022.
{``Artificial Intelligence and Systemic Risk.''} \emph{Journal of
Banking \& Finance} 140: 106290.

\bibitem[\citeproctext]{ref-davis2015commonsense}
Davis, Ernest, and Gary Marcus. 2015. {``Commonsense Reasoning and
Commonsense Knowledge in Artificial Intelligence.''}
\emph{Communications of the ACM} 58 (9): 92--103.

\bibitem[\citeproctext]{ref-dean2023experimental}
Dean, Mark, and Nathaniel Neligh. 2023. {``Experimental Tests of
Rational Inattention.''} \emph{Journal of Political Economy} 131 (12):
3415--61.

\bibitem[\citeproctext]{ref-debreu1984economic}
Debreu, Gerard. 1984. {``Economic Theory in the Mathematical Mode.''}
\emph{The American Economic Review} 74 (3): 267--78.

\bibitem[\citeproctext]{ref-dziri2024faith}
Dziri, Nouha, Ximing Lu, Melanie Sclar, Xiang Lorraine Li, Liwei Jiang,
Bill Yuchen Lin, Sean Welleck, et al. 2024. {``Faith and Fate: Limits of
Transformers on Compositionality.''} \emph{Advances in Neural
Information Processing Systems} 36.

\bibitem[\citeproctext]{ref-ferrario2022eliciting}
Ferrario, Beatrice, and Stefanie Stantcheva. 2022. {``Eliciting People's
First-Order Concerns: Text Analysis of Open-Ended Survey Questions.''}
In \emph{AEA Papers and Proceedings}, 112:163--69. American Economic
Association 2014 Broadway, Suite 305, Nashville, TN 37203.

\bibitem[\citeproctext]{ref-mighty1994fallen}
Gans, Joshua S., and George B. Shepherd. 1994. {``How Are the Mighty
Fallen: Rejected Classic Articles by Leading Economists.''}
\emph{Journal of Economic Perspectives} 8 (1): 165--79.
\url{https://doi.org/10.1257/jep.8.1.165}.

\bibitem[\citeproctext]{ref-rationality2023gilboa}
Gilboa, Itzhak, Stefania Minardi, and Fan Wang. 2023. {``{Schumpeter
Lecture 2023: Rationality and Zero Risk}.''} \emph{Journal of the
European Economic Association} 22 (1): 1--33.
\url{https://doi.org/10.1093/jeea/jvad071}.

\bibitem[\citeproctext]{ref-gilboa2014analogies}
Gilboa, Itzhak, Andrew Postlewaite, Larry Samuelson, and David
Schmeidler. 2014. {``{Economic Models as Analogies}.''} \emph{The
Economic Journal} 124 (578): F513--33.
\url{https://doi.org/10.1111/ecoj.12128}.

\bibitem[\citeproctext]{ref-theory2022gilboa}
---------. 2022. {``Economic Theory: Economics, Methods and
Methodology.''} \emph{Revue Économique} 73 (6): pp. 897--920.
\url{https://www.jstor.org/stable/48714515}.

\bibitem[\citeproctext]{ref-GILBOA20101757}
Gilboa, Itzhak, and David Schmeidler. 2010. {``Simplicity and
Likelihood: An Axiomatic Approach.''} \emph{Journal of Economic Theory}
145 (5): 1757--75.
https://doi.org/\url{https://doi.org/10.1016/j.jet.2010.03.010}.

\bibitem[\citeproctext]{ref-HASSABIS2017245}
Hassabis, Demis, Dharshan Kumaran, Christopher Summerfield, and Matthew
Botvinick. 2017. {``Neuroscience-Inspired Artificial Intelligence.''}
\emph{Neuron} 95 (2): 245--58.
https://doi.org/\url{https://doi.org/10.1016/j.neuron.2017.06.011}.

\bibitem[\citeproctext]{ref-hebert2021neighborhood}
Hébert, Benjamin, and Michael Woodford. 2021. {``Neighborhood-Based
Information Costs.''} \emph{American Economic Review} 111 (10):
3225--55. \url{https://doi.org/10.1257/aer.20200154}.

\bibitem[\citeproctext]{ref-heckman2023econometric}
Heckman, James J, and Rodrigo Pinto. 2023. {``Econometric Causality: The
Central Role of Thought Experiments.''} National Bureau of Economic
Research.

\bibitem[\citeproctext]{ref-heckman2015causal}
Heckman, James, and Rodrigo Pinto. 2015. {``Causal Analysis After
Haavelmo.''} \emph{Econometric Theory} 31 (1): 115--51.

\bibitem[\citeproctext]{ref-johnson2015logic}
Johnson-Laird, Philip N, Sangeet S Khemlani, and Geoffrey P Goodwin.
2015. {``Logic, Probability, and Human Reasoning.''} \emph{Trends in
Cognitive Sciences} 19 (4): 201--14.

\bibitem[\citeproctext]{ref-kleinberg1999authoritative}
Kleinberg, Jon. 1999. {``Authoritative Sources in a Hyperlinked
Environment.''} \emph{Journal of the ACM (JACM)} 46 (5): 604--32.

\bibitem[\citeproctext]{ref-kleinberg2000small}
---------. 2000. {``The Small-World Phenomenon: An Algorithmic
Perspective.''} In \emph{Proceedings of the Thirty-Second Annual ACM
Symposium on Theory of Computing}, 163--70.

\bibitem[\citeproctext]{ref-kleinberg2023inversion}
Kleinberg, Jon, Jens Ludwig, Sendhil Mullainathan, and Manish Raghavan.
n.d. {``The Inversion Problem: Why Algorithms Should Infer Mental State
and Not Just Predict Behavior.''} \emph{Perspectives on Psychological
Science}, 17456916231212138.

\bibitem[\citeproctext]{ref-korinek2023gen}
Korinek, Anton. 2023. {``Generative AI for Economic Research: Use Cases
and Implications for Economists.''} \emph{Journal of Economic
Literature} 61 (4): 1281--1317.
\url{https://doi.org/10.1257/jel.20231736}.

\bibitem[\citeproctext]{ref-laban2023you}
Laban, Philippe, Lidiya Murakhovs' ka, Caiming Xiong, and Chien-Sheng
Wu. 2023. {``Are You Sure? Challenging Llms Leads to Performance Drops
in the Flipflop Experiment.''} \emph{arXiv Preprint arXiv:2311.08596}.

\bibitem[\citeproctext]{ref-larochelle2010learning}
Larochelle, Hugo, and Geoffrey E Hinton. 2010. {``Learning to Combine
Foveal Glimpses with a Third-Order Boltzmann Machine.''} \emph{Advances
in Neural Information Processing Systems} 23.

\bibitem[\citeproctext]{ref-lecun2022path}
LeCun, Yann. 2022. {``A Path Towards Autonomous Machine Intelligence
Version 0.9. 2, 2022-06-27.''} \emph{Open Review} 62.

\bibitem[\citeproctext]{ref-lee2023geometric}
Lee, Yonghyeon. 2023. {``A Geometric Perspective on Autoencoders.''}
\emph{arXiv Preprint arXiv:2309.08247}.

\bibitem[\citeproctext]{ref-levesque2012winograd}
Levesque, Hector, Ernest Davis, and Leora Morgenstern. 2012. {``The
Winograd Schema Challenge.''} In \emph{Thirteenth International
Conference on the Principles of Knowledge Representation and Reasoning}.

\bibitem[\citeproctext]{ref-lewbel2019identification}
Lewbel, Arthur. 2019. {``The Identification Zoo: Meanings of
Identification in Econometrics.''} \emph{Journal of Economic Literature}
57 (4): 835--903.

\bibitem[\citeproctext]{ref-loh2014efficient}
Loh, Lay Kuan, and Mihovil Bartulovic. 2014. {``Efficient Coding
Hypothesis and an Introduction to Information Theory.''} \emph{Mimeo}.

\bibitem[\citeproctext]{ref-lucas1976econometric}
Lucas, Robert E. 1976. {``Econometric Policy Evaluation: A Critique.''}
\emph{Journal of Monetary Economics} 1 (2): 19--46.

\bibitem[\citeproctext]{ref-ludwig2024machine}
Ludwig, Jens, and Sendhil Mullainathan. 2024. {``{Machine Learning as a
Tool for Hypothesis Generation*}.''} \emph{The Quarterly Journal of
Economics}, January, qjad055. \url{https://doi.org/10.1093/qje/qjad055}.

\bibitem[\citeproctext]{ref-mahowald2023dissociating}
Mahowald, Kyle, Anna A. Ivanova, Idan A. Blank, Nancy Kanwisher, Joshua
B. Tenenbaum, and Evelina Fedorenko. 2023. {``Dissociating Language and
Thought in Large Language Models.''}
\url{https://arxiv.org/abs/2301.06627}.

\bibitem[\citeproctext]{ref-marschak1944random}
Marschak, Jacob, and William H Andrews. 1944. {``Random Simultaneous
Equations and the Theory of Production.''} \emph{Econometrica, Journal
of the Econometric Society}, 143--205.

\bibitem[\citeproctext]{ref-mei2024turing}
Mei, Qiaozhu, Yutong Xie, Walter Yuan, and Matthew O. Jackson. 2024.
{``A Turing Test of Whether AI Chatbots Are Behaviorally Similar to
Humans.''} \emph{Proceedings of the National Academy of Sciences} 121
(9): e2313925121. \url{https://doi.org/10.1073/pnas.2313925121}.

\bibitem[\citeproctext]{ref-mitchell2023debate}
Mitchell, Melanie, and David C. Krakauer. 2023. {``The Debate over
Understanding in AI's Large Language Models.''} \emph{Proceedings of the
National Academy of Sciences} 120 (13): e2215907120.
\url{https://doi.org/10.1073/pnas.2215907120}.

\bibitem[\citeproctext]{ref-mnih2014recurrent}
Mnih, Volodymyr, Nicolas Heess, Alex Graves, et al. 2014. {``Recurrent
Models of Visual Attention.''} \emph{Advances in Neural Information
Processing Systems} 27.

\bibitem[\citeproctext]{ref-MOLINARI2020355}
Molinari, Francesca. 2020. {``Chapter 5 - Microeconometrics with Partial
Identification⋆⋆i Thank Don Andrews, Isaiah Andrews, Levon Barseghyan,
Federico Bugni, Ivan Canay, Joachim Freyberger, Hiroaki Kaido, Toru
Kitagawa, Chuck Manski, Rosa Matzkin, Ilya Molchanov, Áureo de Paula,
Jack Porter, Seth Richards-Shubik, Adam Rosen, Shuyang Sheng, Jörg
Stoye, Elie Tamer, Matthew Thirkettle, and Participants to the 2017
Handbook of Econometrics Conference, for Helpful Comments, and the
National Science Foundation for Financial Support Through Grants
SES-1824375 and SES-1824448. I Am Grateful to Louis Liu and Yibo Sun for
Research Assistance Supported by the Robert s. Hatfield Fund for
Economic Education at Cornell University. Part of This Research Was
Carried Out During My Sabbatical Leave at the Department of Economics at
Duke University, Whose Hospitality i Gratefully Acknowledge.''} In
\emph{Handbook of Econometrics, Volume 7A}, edited by Steven N. Durlauf,
Lars Peter Hansen, James J. Heckman, and Rosa L. Matzkin, 7:355--486.
Handbook of Econometrics. Elsevier.
https://doi.org/\url{https://doi.org/10.1016/bs.hoe.2020.05.002}.

\bibitem[\citeproctext]{ref-olshausen1996natural}
Olshausen, Bruno A, and David J Field. 1996. {``Natural Image Statistics
and Efficient Coding.''} \emph{Network: Computation in Neural Systems} 7
(2): 333.

\bibitem[\citeproctext]{ref-parkes2015economic}
Parkes, David C, and Michael P Wellman. 2015. {``Economic Reasoning and
Artificial Intelligence.''} \emph{Science} 349 (6245): 267--72.

\bibitem[\citeproctext]{ref-perez2024testing}
Perez-Cruz, Fernando, and Hyun Song Shin. 2024. {``Testing the Cognitive
Limits of Large Language Models.''} Bank for International Settlements.

\bibitem[\citeproctext]{ref-intrinsic2021}
Pope, Phillip, Chen Zhu, Ahmed Abdelkader, Micah Goldblum, and Tom
Goldstein. 2021. {``The Intrinsic Dimension of Images and Its Impact on
Learning.''} \emph{CoRR} abs/2104.08894.
\url{https://arxiv.org/abs/2104.08894}.

\bibitem[\citeproctext]{ref-pribram1953patterns}
Pribram, Karl. 1953. {``Patterns of Economic Reasoning.''} \emph{The
American Economic Review} 43 (2): 243--58.

\bibitem[\citeproctext]{ref-prystawski2024think}
Prystawski, Ben, Michael Li, and Noah Goodman. 2024. {``Why Think Step
by Step? Reasoning Emerges from the Locality of Experience.''}
\emph{Advances in Neural Information Processing Systems} 36.

\bibitem[\citeproctext]{ref-robbins1932essay}
Robbins, Lionel. 1932. \emph{An Essay on the Nature and Significance of
Economic Science}. Macmillan; Co., Limited.

\bibitem[\citeproctext]{ref-sap2019socialiqa}
Sap, Maarten, Hannah Rashkin, Derek Chen, Ronan LeBras, and Yejin Choi.
2019. {``Socialiqa: Commonsense Reasoning about Social Interactions.''}
\emph{arXiv Preprint arXiv:1904.09728}.

\bibitem[\citeproctext]{ref-schaeffer2024emergent}
Schaeffer, Rylan, Brando Miranda, and Sanmi Koyejo. 2024. {``Are
Emergent Abilities of Large Language Models a Mirage?''} \emph{Advances
in Neural Information Processing Systems} 36.

\bibitem[\citeproctext]{ref-schmitt2019sherliic}
Schmitt, Martin, and Hinrich Schütze. 2019. {``SherLIiC: A Typed
Event-Focused Lexical Inference Benchmark for Evaluating Natural
Language Inference.''} \emph{arXiv Preprint arXiv:1906.01393}.

\bibitem[\citeproctext]{ref-searle1980minds}
Searle, John R. 1980. {``Minds, Brains, and Programs.''}
\emph{Behavioral and Brain Sciences} 3 (3): 417--24.

\bibitem[\citeproctext]{ref-shannon1948mathematical}
Shannon, Claude Elwood. 1948. {``A Mathematical Theory of
Communication.''} \emph{The Bell System Technical Journal} 27 (3):
379--423.

\bibitem[\citeproctext]{ref-SIMS2003665}
Sims, Christopher A. 2003. {``Implications of Rational Inattention.''}
\emph{Journal of Monetary Economics} 50 (3): 665--90.
https://doi.org/\url{https://doi.org/10.1016/S0304-3932(03)00029-1}.

\bibitem[\citeproctext]{ref-srivastava2022beyond}
Srivastava, Aarohi, Abhinav Rastogi, Abhishek Rao, Abu Awal Md Shoeb,
Abubakar Abid, Adam Fisch, Adam R Brown, et al. 2022. {``Beyond the
Imitation Game: Quantifying and Extrapolating the Capabilities of
Language Models.''} \emph{arXiv Preprint arXiv:2206.04615}.

\bibitem[\citeproctext]{ref-stantcheva2023run}
Stantcheva, Stefanie. 2023. {``How to Run Surveys: A Guide to Creating
Your Own Identifying Variation and Revealing the Invisible.''}
\emph{Annual Review of Economics} 15: 205--34.

\bibitem[\citeproctext]{ref-storks2020recent}
Storks, Shane, Qiaozi Gao, and Joyce Y. Chai. 2020. {``Recent Advances
in Natural Language Inference: A Survey of Benchmarks, Resources, and
Approaches.''} \url{https://arxiv.org/abs/1904.01172}.

\bibitem[\citeproctext]{ref-tenenbaum2000global}
Tenenbaum, Joshua B, Vin de Silva, and John C Langford. 2000. {``A
Global Geometric Framework for Nonlinear Dimensionality Reduction.''}
\emph{Science} 290 (5500): 2319--23.

\bibitem[\citeproctext]{ref-vaswani2023attention}
Vaswani, Ashish, Noam Shazeer, Niki Parmar, Jakob Uszkoreit, Llion
Jones, Aidan N. Gomez, Lukasz Kaiser, and Illia Polosukhin. 2023.
{``Attention Is All You Need.''} \url{https://arxiv.org/abs/1706.03762}.

\bibitem[\citeproctext]{ref-walton2014abductive}
Walton, Douglas. 2014. \emph{Abductive Reasoning}. University of Alabama
Press.

\bibitem[\citeproctext]{ref-wei2022emergent}
Wei, Jason, Yi Tay, Rishi Bommasani, Colin Raffel, Barret Zoph,
Sebastian Borgeaud, Dani Yogatama, et al. 2022. {``Emergent Abilities of
Large Language Models.''} \emph{arXiv Preprint arXiv:2206.07682}.

\bibitem[\citeproctext]{ref-wei2022chain}
Wei, Jason, Xuezhi Wang, Dale Schuurmans, Maarten Bosma, Fei Xia, Ed
Chi, Quoc V Le, Denny Zhou, et al. 2022. {``Chain-of-Thought Prompting
Elicits Reasoning in Large Language Models.''} \emph{Advances in Neural
Information Processing Systems} 35: 24824--37.

\bibitem[\citeproctext]{ref-xie2023ask}
Xie, Qiming, Zengzhi Wang, Yi Feng, and Rui Xia. 2023. {``Ask Again,
Then Fail: Large Language Models' Vacillations in Judgement.''}
\emph{arXiv Preprint arXiv:2310.02174}.

\bibitem[\citeproctext]{ref-yang2023harnessing}
Yang, Jingfeng, Hongye Jin, Ruixiang Tang, Xiaotian Han, Qizhang Feng,
Haoming Jiang, Bing Yin, and Xia Hu. 2023. {``Harnessing the Power of
LLMs in Practice: A Survey on Chatgpt and Beyond.''} \emph{arXiv
Preprint arXiv:2304.13712}.

\bibitem[\citeproctext]{ref-zaslavsky2018efficient}
Zaslavsky, Noga, Charles Kemp, Terry Regier, and Naftali Tishby. 2018.
{``Efficient Compression in Color Naming and Its Evolution.''}
\emph{Proceedings of the National Academy of Sciences} 115 (31):
7937--42.

\bibitem[\citeproctext]{ref-zellers2018swagaf}
Zellers, Rowan, Yonatan Bisk, Roy Schwartz, and Yejin Choi. 2018.
{``SWAG: A Large-Scale Adversarial Dataset for Grounded Commonsense
Inference.''} In \emph{Proceedings of the 2018 Conference on Empirical
Methods in Natural Language Processing (EMNLP)}.

\bibitem[\citeproctext]{ref-zellers2019hellaswag}
Zellers, Rowan, Ari Holtzman, Yonatan Bisk, Ali Farhadi, and Yejin Choi.
2019. {``Hellaswag: Can a Machine Really Finish Your Sentence?''}
\emph{arXiv Preprint arXiv:1905.07830}.

\end{CSLReferences}



\end{document}
