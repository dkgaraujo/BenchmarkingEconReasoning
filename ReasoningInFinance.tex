% Options for packages loaded elsewhere
\PassOptionsToPackage{unicode}{hyperref}
\PassOptionsToPackage{hyphens}{url}
\PassOptionsToPackage{dvipsnames,svgnames,x11names}{xcolor}
%
\documentclass[
]{article}

\usepackage{amsmath,amssymb}
\usepackage{iftex}
\ifPDFTeX
  \usepackage[T1]{fontenc}
  \usepackage[utf8]{inputenc}
  \usepackage{textcomp} % provide euro and other symbols
\else % if luatex or xetex
  \usepackage{unicode-math}
  \defaultfontfeatures{Scale=MatchLowercase}
  \defaultfontfeatures[\rmfamily]{Ligatures=TeX,Scale=1}
\fi
\usepackage{lmodern}
\ifPDFTeX\else  
    % xetex/luatex font selection
\fi
% Use upquote if available, for straight quotes in verbatim environments
\IfFileExists{upquote.sty}{\usepackage{upquote}}{}
\IfFileExists{microtype.sty}{% use microtype if available
  \usepackage[]{microtype}
  \UseMicrotypeSet[protrusion]{basicmath} % disable protrusion for tt fonts
}{}
\makeatletter
\@ifundefined{KOMAClassName}{% if non-KOMA class
  \IfFileExists{parskip.sty}{%
    \usepackage{parskip}
  }{% else
    \setlength{\parindent}{0pt}
    \setlength{\parskip}{6pt plus 2pt minus 1pt}}
}{% if KOMA class
  \KOMAoptions{parskip=half}}
\makeatother
\usepackage{xcolor}
\setlength{\emergencystretch}{3em} % prevent overfull lines
\setcounter{secnumdepth}{5}
% Make \paragraph and \subparagraph free-standing
\ifx\paragraph\undefined\else
  \let\oldparagraph\paragraph
  \renewcommand{\paragraph}[1]{\oldparagraph{#1}\mbox{}}
\fi
\ifx\subparagraph\undefined\else
  \let\oldsubparagraph\subparagraph
  \renewcommand{\subparagraph}[1]{\oldsubparagraph{#1}\mbox{}}
\fi


\providecommand{\tightlist}{%
  \setlength{\itemsep}{0pt}\setlength{\parskip}{0pt}}\usepackage{longtable,booktabs,array}
\usepackage{calc} % for calculating minipage widths
% Correct order of tables after \paragraph or \subparagraph
\usepackage{etoolbox}
\makeatletter
\patchcmd\longtable{\par}{\if@noskipsec\mbox{}\fi\par}{}{}
\makeatother
% Allow footnotes in longtable head/foot
\IfFileExists{footnotehyper.sty}{\usepackage{footnotehyper}}{\usepackage{footnote}}
\makesavenoteenv{longtable}
\usepackage{graphicx}
\makeatletter
\def\maxwidth{\ifdim\Gin@nat@width>\linewidth\linewidth\else\Gin@nat@width\fi}
\def\maxheight{\ifdim\Gin@nat@height>\textheight\textheight\else\Gin@nat@height\fi}
\makeatother
% Scale images if necessary, so that they will not overflow the page
% margins by default, and it is still possible to overwrite the defaults
% using explicit options in \includegraphics[width, height, ...]{}
\setkeys{Gin}{width=\maxwidth,height=\maxheight,keepaspectratio}
% Set default figure placement to htbp
\makeatletter
\def\fps@figure{htbp}
\makeatother
% definitions for citeproc citations
\NewDocumentCommand\citeproctext{}{}
\NewDocumentCommand\citeproc{mm}{%
  \begingroup\def\citeproctext{#2}\cite{#1}\endgroup}
\makeatletter
 % allow citations to break across lines
 \let\@cite@ofmt\@firstofone
 % avoid brackets around text for \cite:
 \def\@biblabel#1{}
 \def\@cite#1#2{{#1\if@tempswa , #2\fi}}
\makeatother
\newlength{\cslhangindent}
\setlength{\cslhangindent}{1.5em}
\newlength{\csllabelwidth}
\setlength{\csllabelwidth}{3em}
\newenvironment{CSLReferences}[2] % #1 hanging-indent, #2 entry-spacing
 {\begin{list}{}{%
  \setlength{\itemindent}{0pt}
  \setlength{\leftmargin}{0pt}
  \setlength{\parsep}{0pt}
  % turn on hanging indent if param 1 is 1
  \ifodd #1
   \setlength{\leftmargin}{\cslhangindent}
   \setlength{\itemindent}{-1\cslhangindent}
  \fi
  % set entry spacing
  \setlength{\itemsep}{#2\baselineskip}}}
 {\end{list}}
\usepackage{calc}
\newcommand{\CSLBlock}[1]{\hfill\break\parbox[t]{\linewidth}{\strut\ignorespaces#1\strut}}
\newcommand{\CSLLeftMargin}[1]{\parbox[t]{\csllabelwidth}{\strut#1\strut}}
\newcommand{\CSLRightInline}[1]{\parbox[t]{\linewidth - \csllabelwidth}{\strut#1\strut}}
\newcommand{\CSLIndent}[1]{\hspace{\cslhangindent}#1}

\usepackage{algpseudocode}
\usepackage{algorithm}
\usepackage[noblocks]
{authblk}
\renewcommand*{\Authsep}{, }
\renewcommand*{\Authand}{, }
\renewcommand*{\Authands}{, }
\renewcommand\Affilfont{\small}
\DeclareMathOperator*{\argmin}{arg\,min}
\makeatletter
\@ifpackageloaded{caption}{}{\usepackage{caption}}
\AtBeginDocument{%
\ifdefined\contentsname
  \renewcommand*\contentsname{Table of contents}
\else
  \newcommand\contentsname{Table of contents}
\fi
\ifdefined\listfigurename
  \renewcommand*\listfigurename{List of Figures}
\else
  \newcommand\listfigurename{List of Figures}
\fi
\ifdefined\listtablename
  \renewcommand*\listtablename{List of Tables}
\else
  \newcommand\listtablename{List of Tables}
\fi
\ifdefined\figurename
  \renewcommand*\figurename{Figure}
\else
  \newcommand\figurename{Figure}
\fi
\ifdefined\tablename
  \renewcommand*\tablename{Table}
\else
  \newcommand\tablename{Table}
\fi
}
\@ifpackageloaded{float}{}{\usepackage{float}}
\floatstyle{ruled}
\@ifundefined{c@chapter}{\newfloat{codelisting}{h}{lop}}{\newfloat{codelisting}{h}{lop}[chapter]}
\floatname{codelisting}{Listing}
\newcommand*\listoflistings{\listof{codelisting}{List of Listings}}
\usepackage{amsthm}
\theoremstyle{definition}
\newtheorem{definition}{Definition}[section]
\theoremstyle{remark}
\AtBeginDocument{\renewcommand*{\proofname}{Proof}}
\newtheorem*{remark}{Remark}
\newtheorem*{solution}{Solution}
\newtheorem{refremark}{Remark}[section]
\newtheorem{refsolution}{Solution}[section]
\makeatother
\makeatletter
\makeatother
\makeatletter
\@ifpackageloaded{caption}{}{\usepackage{caption}}
\@ifpackageloaded{subcaption}{}{\usepackage{subcaption}}
\makeatother
\ifLuaTeX
  \usepackage{selnolig}  % disable illegal ligatures
\fi
\usepackage{bookmark}

\IfFileExists{xurl.sty}{\usepackage{xurl}}{} % add URL line breaks if available
\urlstyle{same} % disable monospaced font for URLs
\hypersetup{
  pdftitle={Imperfect reasoning abilities of artificial intelligence models and asset prices (Early stage work)},
  pdfauthor={Douglas K. G. Araujo},
  colorlinks=true,
  linkcolor={blue},
  filecolor={Maroon},
  citecolor={Blue},
  urlcolor={Blue},
  pdfcreator={LaTeX via pandoc}}

\title{Imperfect reasoning abilities of artificial intelligence models
and asset prices (Early stage work)\thanks{This work represents my
opinion and not necessarily that of the BIS.}}


  \author{Douglas K. G. Araujo}
            \affil{%
                  Bank for International
Settlements, douglas.araujo@bis.org
              }
      
\date{}
\begin{document}
\maketitle
\begin{abstract}
Market participants choose the level of costly deployment of artificial
intelligence models, such as large language models, to better extract
signal about fundamentals from a noisy and multidimensional common data
space. The monotonically increasing technology frontier evolves over
time but actual adoption depends on the use of scarce resources that
influence costs, mimicking constraints in human resources and parallel
computing. Participants make decisions based on a `double global game'
with simultaneous higher-order beliefs about asset prices and technology
adoption by peers. Similar to the traditional case with assets,
participants overinvest in AI because they think others might be doing
so, etc. But while this overinvestment in AI reduces the noise about
fundamentals, the higher-order beliefs that everyone else also observes
lower noise ends up leading to the same theme of overinvestment in the
asset as well. If the technology envelope does not include a properly
reasoning AI model, which is able to robustly ignore noise about
fundamentals, the equilibrium outcome is a self-reinforcing
overinvestment both in the asset and in the AI-related resources. In
contrast, the emergence of a reasoning model brings the noise to zero
and agents coordinate perfectly. JEL codes: C45, C69, C88, C59.
\end{abstract}

\section{Introduction}\label{introduction}

The increasing capabilities of large language models (LLM) and other
artificial intelligence (AI) models create expectations that they can be
useful for forecasting and trading (eg, Lopez-Lira and Tang (2023)). But
if all market participants expect others to improve their own
signal-to-noise ratio by adopting a common technology, a coordination
situation with higher-order beliefs similar to the canonical global
games arises, in conjunction with the original coordination issue. This
paper explores the equilibrium outcome when such a technology evolves
according to availability of resources that are also deployed at a cost
by market participants to improve their signal.

Market participants choose the level of costly deployment of artificial
intelligence models, such as large language models, to better extract
signal about fundamentals from a noisy and multidimensional common data
space. The monotonically increasing technology frontier evolves over
time but actual adoption depends on the use of scarce resources that
influence costs, mimicking constraints in human resources and parallel
computing. Participants make decisions based on a `double global game'
with simultaneous higher-order beliefs about asset prices and technology
adoption by peers. Similar to the traditional case with assets,
participants overinvest in AI because they think others might be doing
so, etc. But while this overinvestment in AI reduces the noise about
fundamentals, the higher-order beliefs that everyone else also observes
lower noise ends up leading to the same theme of overinvestment in the
asset as well. If the technology envelope does not include a properly
reasoning AI model, which is able to robustly ignore noise about
fundamentals, the equilibrium outcome is a self-reinforcing
overinvestment both in the asset and in the AI-related resources. In
contrast, the emergence of a reasoning model brings the noise to zero
and agents coordinate perfectly.

\section{Literature}\label{literature}

This work relates to the literatures on

\textbf{Coordination with information acquisition}. Angeletos and Lian
(2016), Szkup and Trevino (2015b), Szkup and Trevino (2015a), Szkup and
Trevino (2021). Reshidi et al. (2021) study the individual and
collective information acquisition. Technology adoption canonical model
D. Frankel and Pauzner (2000). Private aquisition of information
(processing), Hellwig and Veldkamp (2009) and Colombo, Femminis, and
Pavan (2014). In contrast with that literature, here the AI technology
frontier also develops endogenously and responds (negatively) to the
resource take-up from technology adoption\ldots{}

\textbf{AI in finance, including risks.} Danielsson, Macrae, and
Uthemann (2022).

\section{Model}\label{model}

The model draws from the setup in Szkup and Trevino (2021), with
important additions. First, the cost of technology adoption is subject
to a flat supply curve, resulting in cost changes that introduce another
strategic complementarity. Second, the technology itself evolves over
time, and responds to factor prices as well. And third, the process by
which AI improves the signal is laid out in more detail to highlight the
cases where the technology can absorb more text but cannot yet
\emph{reason} about it.

\subsection{Setup}\label{setup}

The simplest version of the model is as follows.

The state of economic fundamentals is a random variable with normal
distribution \(\theta \sim N(\mu_\theta, \sigma_\theta^2)\). \(\theta\)
is only observed indirectly by each investor \(i\) as a noisy signal,
\(x_i = \theta + \sigma_i \epsilon_i\). An AI technology \(\alpha > 0\)
uses \(R > 0\) finite specialised resources (eg, AI scientists, data
engineers, graphics processing units chips - GPUs, etc) to improve the
precision of the signal of the existing data: with
\(\alpha_R' > 0, \alpha_R'' < 0\), and each individual precision defined
as \(\sigma_i = \sigma / \alpha_i(R)\).

However, unless the AI model can actually reason (as in Araujo, 2024;
described further below), very precise signals (low \(\sigma_i\)) also
come with the risk of belieavable misinterpretation (``hallucinations'')
or other forms of ``silent mistakes'', which bias the investor's
perception about fundamentals. Such a problem is of course compounded by
the high confidence the investor has in the signal given its acquired
low \(\sigma_i\). This is modelled through a \emph{reasoning filter}
\(\phi\) (see Section~\ref{sec-reasoning}). This function is the
identity function if the AI cannot reason and 0 if it can reason. Having
\(\eta \sim N(0, \sigma_\eta)\) as the baseline level of
noise\footnote{Reflecting, for example, technology frictions in the
  production and dissemination of data, or more fundamentally even the
  sparsity in the actual signal from the manifold hypothesis.} for all
private signals, then
\(\epsilon_i = \eta + \phi(e^{\lambda \alpha_i(R)})\) (\(\lambda\) is an
inconsequential scaling constant). Putting all of this together, the
private signal about the fundamentals is:

\begin{equation}\phantomsection\label{eq-privatesignals}{
x_i = \theta + (\eta + \phi(e^{\lambda \alpha_i(R)})) \sigma / \alpha_i(R)
}\end{equation}

\begin{itemize}
\tightlist
\item
  Two ex ante identical investors, \(i \in \{1,2\}\) choose how much to
  invest in AI
\end{itemize}

\subsection{Equilibrium in financial and AI
investments}\label{equilibrium-in-financial-and-ai-investments}

After the decisions to invest in AI is taken, the second stage involves
an investment in the financial asset. This step resembles a traditional
global game (Morris and Shin 2003), but where players choose the
precision of their own information (Yang 2015, Szkup and Trevino
(2021)).

\subsection{Tentative calculations}\label{tentative-calculations}

First, note there is exactly one value of \(\alpha_i\) that results in
an unbiased private signal.

\begin{definition}[AI use with minimal bias in private
signal]\protect\hypertarget{def-alphainvestunbiased}{}\label{def-alphainvestunbiased}

There is only one specific value \(\alpha^*\) for which the private
signal is the closest to \(\theta\) in expectation. The subscript is
irrelevant because all investors are ex ante similar.

\end{definition}

See Definition~\ref{def-alphainvestunbiased}.

\begin{proof}
The first order condition only holds for one value of \(\alpha_i > 0\).
Isolating \(\alpha^*\) in the numerator to the left side yields \[
\frac{d}{d \alpha^*} (\eta + \phi(e^{\lambda \alpha^*})) \sigma / \alpha^* = 0,
\]

which can be manipulated to

\[
\frac{\lambda e^{\lambda \alpha^*} \sigma}{\alpha} - \frac{e^{\lambda \alpha^*}\sigma}{(\alpha^*)^2} = \frac{\eta \sigma}{(\alpha^*)^2}.
\]

Multiplying both sides by \((\alpha^*)^2\) obtains

\[
\lambda \alpha^* e^{\lambda \alpha^*} \sigma - e^{\lambda \alpha^*}\sigma = \eta \sigma,
\]

which is the same as

\[
e^{\lambda \alpha^*} (\lambda \alpha^* - 1)= \eta,
\]

and thus the only possible solution is \(\alpha^* = 1/\lambda\).

The second order condition obtained by substituting this equality above
is positive, confirming that \(\alpha^*\) minimises the bias on the
signal:

\[
\frac{d^2}{d (1/\lambda)^2} (\eta + \phi(e)) \sigma \lambda > 0
\]
\end{proof}

The problem is, higher levels of \(R\) monotonically increase precision
but there is only one level of \(R\) that minimises bias in information.

\section{Reasoning}\label{sec-reasoning}

Note that \(\phi\) itself is not dependent on the performance of the AI.
This is in line with compiling evidence that while the increasing power
of AI affords it new abilities, including some that resemble reasoning,
ultimately it is even questionable if a robust for of reasoning can be
obtained by continuing to scale a technique that involves memorisation
rather than actual abstract thinking (LeCun\ldots)

\subsection{Equilibrium}\label{equilibrium}

Equilibrium D. M. Frankel, Morris, and Pauzner (2003).

\section{Statistics}\label{statistics}

Szkup (2020) show that only the direct effect is important.

\section{Conclusions}\label{conclusions}

\section*{References}\label{references}
\addcontentsline{toc}{section}{References}

\phantomsection\label{refs}
\begin{CSLReferences}{1}{0}
\bibitem[\citeproctext]{ref-angeletos2016incomplete}
Angeletos, G-M, and Chen Lian. 2016. {``Incomplete Information in
Macroeconomics: Accommodating Frictions in Coordination.''} In
\emph{Handbook of Macroeconomics}, 2:1065--1240. Elsevier.

\bibitem[\citeproctext]{ref-colombo2014information}
Colombo, Luca, Gianluca Femminis, and Alessandro Pavan. 2014.
{``Information Acquisition and Welfare.''} \emph{The Review of Economic
Studies} 81 (4): 1438--83.

\bibitem[\citeproctext]{ref-danielsson2022artificial}
Danielsson, Jon, Robert Macrae, and Andreas Uthemann. 2022.
{``Artificial Intelligence and Systemic Risk.''} \emph{Journal of
Banking \& Finance} 140: 106290.

\bibitem[\citeproctext]{ref-frankel2003equilibrium}
Frankel, David M, Stephen Morris, and Ady Pauzner. 2003. {``Equilibrium
Selection in Global Games with Strategic Complementarities.''}
\emph{Journal of Economic Theory} 108 (1): 1--44.

\bibitem[\citeproctext]{ref-techadoption}
Frankel, David, and Ady Pauzner. 2000. {``{Resolving Indeterminacy in
Dynamic Settings: The Role of Shocks*}.''} \emph{The Quarterly Journal
of Economics} 115 (1): 285--304.
\url{https://doi.org/10.1162/003355300554746}.

\bibitem[\citeproctext]{ref-hellwig2009knowing}
Hellwig, Christian, and Laura Veldkamp. 2009. {``Knowing What Others
Know: Coordination Motives in Information Acquisition.''} \emph{The
Review of Economic Studies} 76 (1): 223--51.

\bibitem[\citeproctext]{ref-lopez2023can}
Lopez-Lira, Alejandro, and Yuehua Tang. 2023. {``Can Chatgpt Forecast
Stock Price Movements? Return Predictability and Large Language
Models.''} \emph{arXiv Preprint arXiv:2304.07619}.

\bibitem[\citeproctext]{ref-reshidi2021individual}
Reshidi, Pëllumb, Alessandro Lizzeri, Leeat Yariv, Jimmy H Chan, and
Wing Suen. 2021. {``Individual and Collective Information Acquisition:
An Experimental Study.''} National Bureau of Economic Research.

\bibitem[\citeproctext]{ref-szkup2020multiplier}
Szkup, Michal. 2020. {``Multiplier Effect and Comparative Statics in
Global Games of Regime Change.''} \emph{Theoretical Economics} 15 (2):
625--67.

\bibitem[\citeproctext]{ref-szkup2015informationtr}
Szkup, Michal, and Isabel Trevino. 2015a. {``Information Acquisition and
Transparency in Global Games.''} \emph{Journal of Economic Theory} 160:
387--428.

\bibitem[\citeproctext]{ref-szkup2015informationgg}
---------. 2015b. {``Information Acquisition in Global Games of Regime
Change.''} \emph{Journal of Economic Theory} 160: 387--428.

\bibitem[\citeproctext]{ref-szkup2021information}
---------. 2021. {``Information Acquisition and Self-Selection in
Coordination Games.''} mimeo.

\end{CSLReferences}



\end{document}
